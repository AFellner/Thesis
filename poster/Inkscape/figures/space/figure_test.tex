\documentclass[a4paper,11pt,twoside]{memoir}
\usepackage{amsthm}
\usepackage{enumitem,amsmath,amssymb}

\usepackage{url}
\usepackage{hyperref}					% links in pdf
\usepackage{graphicx}            			% Figures
\usepackage{verbatim}            			% Code-Environment
\usepackage[linesnumbered,algochapter,noend,ruled]{algorithm2e} % Algorithm-Environment

\usepackage{pgf}					
\usepackage{tikz}					% tikz graphics
\usetikzlibrary{arrows,automata,positioning}
\usetikzlibrary{fit}
\usepackage{ngerman}
\usepackage[ngerman]{babel}
\usepackage{bibgerm,cite}       % Deutsche Bezeichnungen, Automatisches Zusammenfassen von Literaturstellen
\usepackage[ngerman]{varioref}  % Querverweise
% to use the german charset include cp850 for MS-DOS, ansinew for Windows and latin1 for Linux.
% \usepackage[latin1]{inputenc}

\pagenumbering{gobble}

\usepackage{S:/Users/Andreas/Documents/studium/emcl/Wien/Thesis/latex/drawproof}

\begin{document}

\documentclass[a4paper,11pt]{article}
\usepackage{tikz}
\pagenumbering{gobble}
\begin{document}
\centering
\begin{tikzpicture}[node distance=2.5cm]
	
	\node[draw,very thick,circle,font=\bf](n7){$7$};
	
	\node[draw,very thick,circle,above left of=n7,font=\bf,shade=black](n5){$5$};
	\node[draw,very thick,circle,above right of=n7,font=\bf](n6){$6$};
	
	\node[draw,very thick,circle,above left of=n5,xshift=.5cm,shade=black,font=\bfseries](n1){$1$};
	\node[draw,very thick,circle,above right of=n5,xshift=-.5cm,shade=black,font=\bfseries](n2){$2$};
	
	\node[draw,very thick,circle,above left of=n6,xshift=.5cm,shade=black,font=\bfseries](n3){$3$};
	\node[draw,very thick,circle,above right of=n6,xshift=-.5cm,shade=black,font=\bfseries](n4){$4$};
	
	\draw[->,very thick,cap=round] (n7) -- (n5);
	\draw[->,very thick,cap=round] (n7) to (n6);
	
	\draw[->,very thick,cap=round] (n5) to (n1);
	\draw[->,very thick,cap=round] (n5) to (n2);
	
	\draw[->,very thick,cap=round] (n6) to (n3);
	\draw[->,very thick,cap=round] (n6) to (n4);
	
	
\end{tikzpicture}
\end{document}

%$$
%\begin{array}{l l}
	%\bullet \text{ reflexivity:} & t = t \\
	%\bullet\text{ symmetry:} & t = s \rightarrow s = t \\
	%\bullet\text{ transitivity:} & t_1 = t_2 \wedge \ldots \wedge t_{n-1} = t_n \rightarrow t_1 = t_n \\
	%\bullet\text{ compatibility:} & \bigwedge_{i=1}^{n} t_i = s_i \rightarrow f(t_1,\ldots,t_n) = f(s_1,\ldots, s_n)
%\end{array}
%$$
\end{document}
