\section{Proof Processing}
\label{sec:proofprocessing}

In Section \ref{sec:resolution} the notion of a proof in our sense was defined.
The aim of this work is to make proof processing easier by minimizing proofs in the two measures space and length.
Proof processing could be checking its correctness, manipulating it, like we do in this work extensively, or extracting information, like interpolants and unsat cores, from it.
The following definition makes the notion of proof processing formal.

\begin{definition}[Proof Processing]
\label{def:proof-processing}

Let $\varphi$ be a proof with nodes $V$ and $T$ be an arbitrary set.
A function $f: V \times T \times T \rightarrow T$ is a \emph{processing function} if there is a function $g_f: V \rightarrow T$ such that for every $v \in V$ with $\Premises{v}{\varphi} = \emptyset$ (i.e. $v$ represents an axiom), $g_f(v) = f(v,t_1,t_2)$ for all $\{t_1,t_2\} \subseteq T$.
Let $\mathcal{F}$ be the set of processing functions.
The \emph{apply function} $\ap: V \times \mathcal{F} \rightarrow T$ is defined recursively as follows.
$$
\ap(\n,f) = \Big\{
\begin{array}{ll}
	f(\n,\ap(pr_1,f),\ap(pr_2,f)) &\text{ if } \n \text{ has premises } pr_1 \text{ and } pr_2\\
	g_f(\n) &\text{ otherwise}\\
\end{array}
$$

\emph{Processing a node} $\n$ with some processing function $f$ means computing the value $\ap(\n,f)$.
\emph{Processing a proof} means to process its root node.

\end{definition}

\begin{example}

Checking the correctness of a proof (i.e. checking for the absence of faulty resolution steps) can be done in terms of the following processing function with $T = \{\top,\bot\}$ and $\wedge$ being the usual boolean and-operation.
$$
f(\n,w_1,w_2) = \left\{
\begin{array}{ll}
	\top & \text{ if $\n$ has no premises} \\
	w_1 \wedge w_2 &\text{ if the conclusion of $\n$ is a resolvent}\\
								 &\quad \text{ of the conclusions of its premises} \\
	\bot & \text{ otherwise}
\end{array}
\right.
$$
Processing a proof with this processing function yields $\top$ if and only if the proof is a correct resolution proof.
\qed
\end{example}