
In this work we presented two methods for proof compression and their theoretical foundations.
We implemented the methods and evaluated them extensively to show their usefulness.
Furthermore, we investigated the complexity of the underlying problems and in the case of explanation production proved the complexity, which is a new result for complexity theory.
For both methods we highlighted possible future work, which could improve the presented methods further.

The method for compressing proofs in length shows good results both in compression rate and speed.
It can compete with previously proposed compression algorithms in that regard.
The compression rate grows with the proof size and since compressing proofs is especially interesting for huge proofs, this puts an even better light on the results measured.
Furthermore, the novel congruence closure algorithm can be used in other applications.
It is especially interesting when immutable data structures are desired, which is almost always the case in functional programming languages.

Two algorithms together with a variety of heuristics for compressing proofs with respect to space have been conceived. 
The experimental evaluation clearly shows that the so-called Bottom-Up algorithm is faster and compresses more than the more natural, straightforward and simple Top-Down algorithm. 
The best performances are achieved with the simplest heuristics (i.e. Last Child and Number of Children). 
More sophisticated heuristics provided little extra compression but cost a high price in execution time. 

The key contributions of this work are two novel proof compression methods and the proof of NP-completeness of the shortest explanation decision problem.