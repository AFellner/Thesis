\section*{Resolution extended with equality}

The main theme of this work is resolution proof compression.
Therefore we need to express equality reasoning in terms of resolution proofs.
We extend the resolution calculus, presented in Section \ref{sec:resolution}, with the axioms of equality.
First we extend the notions of atoms, literals and clauses

\begin{definition}[Equality atom, literal and clause]

Let $\mathcal{T}$ be a set of terms and let $V$ be a finite set of propositional variables.
The set of \emph{equality atoms} is defined as $V \cup \mathcal{T} \times \mathcal{T}$.
An \emph{equality literal} is an equality atom $e$ or a negated equality atom $\neg e$.
An \emph{equality clause} is a set of equality literals.
For an equality clause $C$, we call the sets of equations $pos(C) := \{(u,v) \mid u = v \in C\}$ the \emph{positive part} and $neg(C) := \{(u,v) \mid u \neg v \in C\}$ the \emph{negative part} of $C$.

\end{definition}

A set of equations can be interpreted as a set of clauses, if every equation is interpreted as the singleton clause containing just the equation itself.
In the context of equality atoms, we write equations $(s,t) \in \mathcal{T} \times \mathcal{T}$ as $s = t$ and $s \neq t$ for its negated version.
As usual, a clause is interpreted as the disjunction of its literals and a set of clauses is interpreted as the conjunction its clauses.\\

From hereon, we restrict our attention to sets of terms that, on top of constants, have at most one function symbol $f$, which is binary.
We justify this restriction in Section \ref{subsec:algorithms_preliminaries}.
The axioms defining congruence relations have to be reflected in our extended resolution proofs.
We achieve this by defining axiom schemas, that can be instantiated with concrete terms.

\begin{definition}[Axioms of Equality]

In the following axioms schemas, the variables $x_1,\ldots,x_n$ are placeholders for terms.
By simultaneously replacing all variables by terms of some set $\mathcal{T}$, one obtains an equality clause, which we call an \emph{instance w.r.t. $\mathcal{T}$} of the respective axiom of equality.

\begin{itemize}
	\item reflexive: $\{x = x\}$
	\item symmetric: $\{x_1 \neq x_2, x_2 = x_1\}$
	\item transitive: $\{x_1 \neq x_2, x_2 \neq x_3, \ldots, x_{n-1} \neq x_n, x_1 = x_n\}$
	\item compatible: $\{x_1 \neq x_3, x_2 \neq x_4, f(x_1,x_2) = f(x_3,x_4)\}$
\end{itemize}

\end{definition}

Next we will define the resolution calculus extended by congruence axioms.

\begin{definition}[Resolution with Equality]

Let $\ell$ be an equality literal and $C_1$, $C_2$ be equality clauses such that $\ell \in C_1$ and $\neg \ell \in C_2$.
The clause $C_1 \setminus \{\ell\} \cup C_2 \setminus \{\neg \ell\}$ is the \emph{resolvent} of $C_1$ and $C_2$ with \emph{pivot} $\ell$.

\noindent Let $F = \{C_1, \ldots, C_n\}$ be a set of clauses and let $E$ be the biggest subset of $F$, such that every clause in $E$ is an equation.
The notion of a \emph{congruence derivation} for $F$ is defined inductively.
\begin{itemize}
	\item $\langle C_1, \ldots, C_n\rangle$ is a congruence derivation for $F$.
	\item If $\langle C_1, \ldots, C_m\rangle$ is a congruence derivation for $F$ then $\langle C_1, \ldots, C_{m+1} \rangle$ is a congruence derivation for $F$ if $C_{m+1}$ is an instance w.r.t. $\mathcal{T}_E$ of an axiom of equality or $C_{m+1}$ is a resolvent of $C_i$ and $C_j$ with $1 \leq i,j \leq m$.
\end{itemize}
A \emph{congruence refutation} is a congruence derivation containing the empty clause.

\noindent Let $D = \langle C_1, \ldots, C_m\rangle$ be a congruence derivation.
The longest subsequence $\langle C_{i_1}, \ldots, C_{i_k}\rangle$ of $D$, such that $C_{i_1}, \ldots, C_{i_k}$ all are instances of axioms of equality, is called the equality reasoning part of $D$.

\end{definition}

Just like resolution derivations, congruence derivations can be visualized as directed acyclic graphs.

\begin{proposition}[Sound- \& Completeness]

Let $E$ be a set of equations and $s,t \in \mathcal{T}_E$, then $E \models s \thickapprox t$ if and only if there is a congruence refutation for $E \cup \{\{ s \neq t\}\}$

\end{proposition}

\begin{proof}

The existence of a congruence refutation in case $E \models s \thickapprox t$ is proven in terms of a proof producing algorithm, presented in Section \ref{sec:sec:proofproduction}.
This algorithm produces a congruence derivation with last clause $\{u_1 \neq v_1,\ldots,u_n \neq v_n, s = t\}$ such that $\{(u_i,v_i) \mid i = 1,\ldots,n\} \subseteq E$.
Clearly this proof can be extended to a congruence derivation for $E \cup \{ s \neq t\}$.

Suppose there is a congruence refutation $\langle C_1, \ldots, C_n \rangle$ for $E \cup \{\{ s \neq t\}\}$.
Clearly we can assume that the equality reasoning part of this derivation is the whole sequence.
Since every clause in $E \cup \{\{ s \neq t\}\}$ is singleton, none of its literals is in the resolvent of such a clause with any other clause.
Therefore we can assume that there is a $m < n$ such that $\{C_1, \ldots, C_m\}$ only contains of instances of equality axioms and recursive resolvents of such clauses.
Furthermore, since the whole sequence is a refutation, we can assume that $C_m$ is such that $neg(C_m) \subseteq E$ and $pos(C_m) = \{(s,t)\}$.

We show by induction on the clause structure, that for every clause $C \in \{C_1,\ldots,C_m\}$ that $pos(C) = \{(u,v)\}$ for some $(u,v) \in \mathcal{T}_E$ and that $neg(C) \models u \thickapprox v$.

\begin{itemize}

\item \textbf{Induction Base:} $C$ is an instance of an equality axiom. Clearly, the positive part contains of some equation $(u,v)$ and $neg(C) \models u \thickapprox v$ follows directly form the definition of congruence closure, and in case of the transitive axiom also from the transitivity of equality.

\item \textbf{Induction Step}

\end{itemize}

The positive part of every instance of an axiom of equality is a singleton.
Therefore also the positive part of every resolvent of such clauses .
Let us denote that positive equality literal 
We show by induction on congruence derivations, that for every clause $C$ it holds that $neg(C) \models $
Let $C$ be an instance of an equality axiom.
We show that $C$
Suppose there is a resolution refutation in 
Every clause in $E \cup \{\{ s \neq t\}\}$

\end{proof}