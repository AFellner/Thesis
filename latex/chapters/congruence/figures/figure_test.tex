\documentclass[a4paper,11pt,twoside]{memoir}
\usepackage{amsthm}
\usepackage{enumitem,amsmath,amssymb}

\usepackage{url}
\usepackage{hyperref}					% links in pdf
\usepackage{graphicx}            			% Figures
\usepackage{verbatim}            			% Code-Environment
\usepackage[linesnumbered,algochapter,noend,ruled]{algorithm2e} % Algorithm-Environment

\usepackage{pgf}					
\usepackage{tikz}					% tikz graphics
\usetikzlibrary{arrows,automata,positioning}
\usetikzlibrary{fit}
\usepackage{ngerman}
\usepackage[ngerman]{babel}
\usepackage{bibgerm,cite}       % Deutsche Bezeichnungen, Automatisches Zusammenfassen von Literaturstellen
\usepackage[ngerman]{varioref}  % Querverweise
% to use the german charset include cp850 for MS-DOS, ansinew for Windows and latin1 for Linux.
% \usepackage[latin1]{inputenc}

\pagenumbering{gobble}

\usepackage{../../../drawproof}

\begin{document}

%
\centering
\begin{tikzpicture}[node distance=2.5cm]
	
	
	%\rootnode;
	
	\proofnode{root}{$\bot$};
	
	\proofnode[above right of=root,yshift=.5cm]{new9}{$f(f(a,b),f(a,a)) = f(f(b,a),f(b,b))$};
	\proofnode[above left of=root,yshift=-.5cm,xshift=-1cm]{new10}{$f(f(a,b),f(a,a)) \neq f(f(b,a),f(b,b))$};
	
	\proofnode[above right of=new9,yshift=-1cm,xshift=1cm]{new8}{$a = b$};

	\proofnode[above left of=new9,yshift=-.5cm]{new5} {$a \neq b, f(f(a,b),f(a,a)) = f(f(b,a),f(b,b))$};
	
	\proofnode[above right of=new5,yshift=.25cm]{n5}{$a \neq b, f(a,a) \neq f(b,b), f(f(a,b),f(a,a)) = f(f(b,a),f(b,b))$};
	\proofnode[above left of=new5,yshift=-.5cm]{n7}{$a\neq b, f(a,a) = f(b,b)$};

	\proofnode[above left of=n5,yshift=.25cm]{n3}{$f(a,b) \neq f(b,a), f(a,a) \neq f(b,b), f(f(a,b),f(a,a)) = f(f(b,a),f(b,b))$};
	\proofnode[above right of=n5,yshift=-.5cm]{n4}{$a \neq b, f(a,b) = f(b,a)$};

	\drawchildren{root}{new9}{new10};
	\drawchildren{new5}{n5}{n7};
	\drawchildren{n5}{n3}{n4};
	
	\drawchildren{new9}{new5}{new8};
	
	\proofnode[above left of=new8,yshift=-.6cm,xshift=.5cm]{new6}{$\neg A$};
	\proofnode[above right of=new8,yshift=-.6cm,xshift=-.5cm]{new7}{$A,a = b$};
	\drawchildren{new8}{new6}{new7};
\end{tikzpicture}


%$$
%\begin{array}{l l}
	%\bullet \text{ reflexivity:} & t = t \\
	%\bullet\text{ symmetry:} & t = s \rightarrow s = t \\
	%\bullet\text{ transitivity:} & t_1 = t_2 \wedge \ldots \wedge t_{n-1} = t_n \rightarrow t_1 = t_n \\
	%\bullet\text{ compatibility:} & \bigwedge_{i=1}^{n} t_i = s_i \rightarrow f(t_1,\ldots,t_n) = f(s_1,\ldots, s_n)
%\end{array}
%$$
\end{document}
