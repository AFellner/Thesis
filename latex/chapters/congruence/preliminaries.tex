\section*{Definitions}

We summarize basic definitions notions that we will use throughout the congruence part of this work.
We start with the usual definition of terms.

\begin{definition}[Terms]

Let $\mathcal{F}$ be a finite set of function symbols and $arity: \mathcal{F} \rightarrow \mathbb{N}$.
A tuple $\Sigma = \langle \mathcal{F}, arity \rangle$ is called a \emph{signature}.
A function symbol with arity zero is called a \emph{constant}, one with arity one is called a \emph{unary} function symbol and one with arity 2 is called \emph{binary}.
For a given signature $\Sigma$, the set of \emph{terms} $\mathcal{T}^{\Sigma}$ is defined inductively.

\begin{align*}
	\mathcal{T}^{\Sigma}_0 &= \{a \in \mathcal{F} \mid arity(a) = 0\}\\
	\mathcal{T}^{\Sigma}_{i+1} &= \{g(t_1,\ldots,t_n) \mid arity(g) = n \text{ and } t_1, \ldots, t_n \in \mathcal{T}_{i}\} \\
	\mathcal{T}^{\Sigma} &= \bigcup_{i\in \mathbb{N}} \mathcal{T}_{i}
\end{align*}

An equation of $\mathcal{T}$ is a tuple of terms, i.e. an element of $\mathcal{T} \times \mathcal{T}$.

\end{definition}

We will ommit the index $\Sigma$, if it is clear from context.
If $E$ is a set of equations then we denote by $\mathcal{T}_E$ the set of terms used in $E$, i.e. the set $\{t \mid \text{ for some } v: (t,v) \in E \text{ or } (v,t) \in E\}$.

\begin{definition}[Congruence Relation]

Let $\mathcal{T}$ be a set of terms.
A relation $R \subset \mathcal{T} \times \mathcal{T}$ is a congruence relation, if has the following four properties:
\begin{itemize}
	\item reflexive: for all $t \in \mathcal{T}: (t,t) \in R$
	\item symmetric: $(s,t) \in R$ implies $(t,s) \in R$
	\item transitive: $(r,s) \in R$ and $(s,t) \in R$ implies $(r,t) \in R$
	\item compatible: $f$ is a n-ary function symbol and for all $i = 1,\ldots,n (t_i,s_i) \in R$ implies $f(t_1,\ldots,t_n),f(s_1,\ldots,s_n) \in R$
\end{itemize}

\end{definition}

Every congruence relation partitions its underlying set of terms $\mathcal{T}$ into congruence classes, such that two terms $(s,t)$ belong to the same class if and only if $(s,t) \in R$.
The relations $\mathcal{T} \times \mathcal{T}$ and $\emptyset$ are trivial congruence relations.

Let $E$ be a set of equations with terms in some set of terms $\mathcal{T}$.
The set $E^* \supseteq E$ is called the congruence closure of $E$, 
if $E^*$ is a congruence relation on $\mathcal{T}$ and for every congruence relation $C$, such that $C \supset E$ follows $C \supseteq E^*$.
It is easily seen that congruence relations are closed under intersection.
Therefore $E^*$ always exists.

We write $E \models s \thickapprox t$ if $(s,t) \in E^*$.

We extend the resolution calculus, presented in Section \ref{sec:resolution}, with the axioms of equality.
Let $\mathcal{Q}_{\mathcal{T}} = \{t_1 = t_2 \mid t_1, t_2 \in \mathcal{T}$ be the set of all equalities for a set of terms $\mathcal{T}$
Let $V$ be a finite set of propositional variables.
The set of equality atoms $\mathcal{E}$ is defined as $V \cup Q_{\mathcal{T}}$
An equality literal $\ell_\mathcal{T}$ is an equality atom $e$ or a negated equality atom $\neg e$.
We will abbreviate write $\neg (t_1 = t_2)$ by $t_1 \neq t_2$.
An equality clause is a set of equality literals.
As usual a clause is interpreted as the disjunction of its literals and a set of clauses is interpreted as the conjunction its clauses.

The axioms of congruence $EqAxioms$ for some set of terms $\mathcal{T}$ is defined as $R \cup S \cup T \cup C$ where

\begin{align*}
	R &= \{ t = t \mid t \in \mathcal{T}\} \\
	S &= \{ t_1 \neq t_2, t_2 = t_1 \mid t_1, t_2 \in \mathcal{T} \} \\
	T &= \{ t_1 \neq t_2, t_2 \neq t_3, t_1 = t_3 \mid t_1, t_2, t_3 \in \mathcal{T} \} \\
	C &= \{ t_1 \neq t_3, t_2 \neq t_4, f(t_1,t_2) = f(t_3,t_4) \mid t_1, t_2, t_3, t_4 \in \mathcal{T} \} 
\end{align*}

Note that every congruence axiom has one positive equality literal.
From now on we will omit the set of terms $\mathcal{T}$ if it is clear from context.

Next we will define the resolution calculus extended by congruence axioms.
Let $\ell$ be an equality literal and $C_1$, $C_2$ be equality clauses such that $\ell \in C_1$ and $\neg \ell \in C_2$.
The clause $C_1 \setminus \{\ell\} \cup C_2 \setminus \{\neg \ell\}$ is the resolvent of $C_1$ and $C_2$ with pivot $\ell$.

Let $F = \{C_1, \ldots, C_n\}$ be a set of clauses.
The notion of a congruence derivation for $F$ is defined inductively.
The sequence $\langle C_1, \ldots, C_n\rangle$ is a congruence derivation for $F$.
If $\langle C_1, \ldots, C_m\rangle$ is a congruence derivation for $F$ then $\langle C_1, \ldots, C_{m+1} \rangle$ is a congruence derivation for $F$ if $C_{m+1} \in EqAxioms$ or $C_{m+1}$ is a resolvent of $C_i$ and $C_j$ with $1 \leq i,j \leq m$.
A congruence derivation containing the empty clause is a congruence refutation.

Let $D = \langle C_1, \ldots, C_m\rangle$ be a congruence derivation.
The longest subsequence $\langle C_{i_1}, \ldots, C_{i_k}\rangle$ of $D$, such that $\{C_{i_1}, \ldots, C_{i_k}\} \subseteq EqAxioms$ is called the equality reasoning part of $D$.


%In Section \ref{sec:algorithms}, we will work with s
{\color{blue} Todo: introduce all used notions}