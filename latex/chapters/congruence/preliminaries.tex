\section*{Definitions}
\ref{sec:congruencedef}
We summarize basic definitions notions that we will use throughout the congruence part of this work.
We start with the usual definition of terms.

\begin{definition}[Terms and Subterms]

Let $\mathcal{F}$ be a finite set of function symbols and $arity: \mathcal{F} \rightarrow \mathbb{N}$.
A tuple $\Sigma = \langle \mathcal{F}, arity \rangle$ is called a \emph{signature}.
A function symbol with arity zero is called a \emph{constant}, one with arity one is called a \emph{unary} function symbol and one with arity 2 is called \emph{binary}.
For a given signature $\Sigma$, the set of \emph{terms} $\mathcal{T}^{\Sigma}$ is defined inductively.

\begin{align*}
	\mathcal{T}^{\Sigma}_0 &= \{a \in \mathcal{F} \mid arity(a) = 0\}\\
	\mathcal{T}^{\Sigma}_{i+1} &= \{g(t_1,\ldots,t_n) \mid arity(g) = n \text{ and } t_1, \ldots, t_n \in \mathcal{T}_{i}\} \\
	\mathcal{T}^{\Sigma} &= \bigcup_{i\in \mathbb{N}} \mathcal{T}_{i}
\end{align*}

\noindent Let $f(t_1,\ldots,t_n) \in \mathcal{T}^{\Sigma}$, then $t_1,\ldots,t_n$ are \emph{direct subterms} of $f(t_1,\ldots,t_n)$.
The \emph{subterm} relation is the reflexive, transitive closure of the direct subterm relation.

\end{definition}

We will omit the index $\Sigma$, if it is clear from context.

\begin{definition}[Equation]

Let $\mathcal{T}^{\Sigma}$ be a set of terms.
An \emph{equation} of $\mathcal{T}^{\Sigma}$ is a tuple of terms, i.e. an element of $\mathcal{T}^{\Sigma} \times \mathcal{T}^{\Sigma}$.

\end{definition}

For a set of equations $E$ we denote by $\mathcal{T}_E$ the set of terms used in $E$, i.e. the set $\{t \mid t \text{ is subterm of some } u \text{ such that for some } v: (u,v) \in E \text{ or } (v,u) \in E\}$.

\begin{definition}[Congruence Relation]

Let $\mathcal{T}$ be a set of terms.
A relation $R \subseteq \mathcal{T} \times \mathcal{T}$ is a congruence relation, if has the following four properties:
\begin{itemize}
	\item reflexive: for all $t \in \mathcal{T}: (t,t) \in R$
	\item symmetric: $(s,t) \in R$ implies $(t,s) \in R$
	\item transitive: $(r,s) \in R$ and $(s,t) \in R$ implies $(r,t) \in R$
	\item compatible: $f$ is a n-ary function symbol and for all $i = 1,\ldots,n (t_i,s_i) \in R$ implies $f(t_1,\ldots,t_n),f(s_1,\ldots,s_n) \in R$
\end{itemize}

\end{definition}

Clearly every congruence relation is also an equivalence relation (which is a reflexive, transitive and symmetric relation).
Therefore every congruence relation partitions its underlying set of terms $\mathcal{T}$ into congruence classes, such that two terms $(s,t)$ belong to the same class if and only if $(s,t) \in R$.
The relations $\emptyset$ and $\mathcal{T} \times \mathcal{T}$ are trivial congruence relations.
In this work we are interested in congruence relations induced by sets of equations.
In other words, we compute the partitioning of the terms such that two terms in the same partition are proven to be equal by the input set of equations.
To this end we define the notion of congruence closure of a set of equations.

\begin{definition}[Congruence Closure]

Let $E$ be a set of equations.
The set $E^* \supseteq E$ is called the \emph{congruence closure} of $E$, 
if $E^*$ is a congruence relation on $\mathcal{T}_E$ and for every congruence relation $C$, such that $C \supset E$ follows $C \supseteq E^*$.
It is easily seen that congruence relations are closed under intersection.
Therefore $E^*$ always exists.

\noindent We write $E \models s \thickapprox t$ if $(s,t) \in E^*$ and say that $E$ is an \emph{explanation} for $s \thickapprox t$.
We call a pair $(s,t)$ in a congruence closure an \emph{equality} and we call an equality of compound terms $f(a,b),f(c,d)$ such that $E \models a \thickapprox c$ and $E \models b \thickapprox d$ a \emph{deduced equality}.

\end{definition}

\begin{proposition}[Properties of the $\models$-relation}
\label{prop:models}
The $\models$-relation is monotone: $E_1 \subset E_2$ and $E_1 \models s \thickapprox t$ implies $E_2 \models s \thickapprox t$ and transitive: $E \models s \thickapprox t$ and $E \cup \{(s,t)\} \models u \thickapprox v$ implies $E \models u \thickapprox v$. 

\end{proposition}

\begin{proof}

Monotonicity follows from the fact that congruence closure of $E_1$ is contained in the congruence closure of $E_2$.
Since the congruence closure of $E^*$ is $E^*$ itself, it follows that $E \models u \thickapprox v$ if and only if $E^* \models u \thickapprox v$.
Since $(s,t) \in E^*$, clearly it is the case that $E^* = (E \cup \{(s,t)\})^*$.
Therefore $E \cup \{(s,t)\} \models u \thickapprox v$ implies $(u,v) \in E^*$, i.e. $E \models u \thickapprox v$ or in other words, the $\models$-relation is transitive.

\end{proof}

The idea of using congruence closure in proof compression is to replace big explanations, w.r.t. cardinality, by smaller ones.