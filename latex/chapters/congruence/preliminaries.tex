\section*{Definitions}

We summarize basic definitions notions that we will use throughout the congruence part of this work.
We start with the usual definition of terms.

\begin{definition}[Terms and Subterms]

Let $\mathcal{F}$ be a finite set of function symbols and $arity: \mathcal{F} \rightarrow \mathbb{N}$.
A tuple $\Sigma = \langle \mathcal{F}, arity \rangle$ is called a \emph{signature}.
A function symbol with arity zero is called a \emph{constant}, one with arity one is called a \emph{unary} function symbol and one with arity 2 is called \emph{binary}.
For a given signature $\Sigma$, the set of \emph{terms} $\mathcal{T}^{\Sigma}$ is defined inductively.

\begin{align*}
	\mathcal{T}^{\Sigma}_0 &= \{a \in \mathcal{F} \mid arity(a) = 0\}\\
	\mathcal{T}^{\Sigma}_{i+1} &= \{g(t_1,\ldots,t_n) \mid arity(g) = n \text{ and } t_1, \ldots, t_n \in \mathcal{T}_{i}\} \\
	\mathcal{T}^{\Sigma} &= \bigcup_{i\in \mathbb{N}} \mathcal{T}_{i}
\end{align*}

\noindent Let $f(t_1,\ldots,t_n) \in \mathcal{T}^{\Sigma}$, then $t_1,\ldots,t_n$ are \emph{direct subterms} of $f(t_1,\ldots,t_n)$.
The \emph{subterm} relation is the reflexive, transitive closure of the direct subterm relation.

\end{definition}

We will omit the index $\Sigma$, if it is clear from context.

\begin{definition}[Equation]

Let $\mathcal{T}^{\Sigma}$ be a set of terms.
An \emph{equation} of $\mathcal{T}^{\Sigma}$ is a tuple of terms, i.e. an element of $\mathcal{T}^{\Sigma} \times \mathcal{T}^{\Sigma}$.

\end{definition}

For a set of equations $E$ we denote by $\mathcal{T}_E$ the set of terms used in $E$, i.e. the set $\{t \mid t \text{ is subterm of some } u \text{ such that for some } v: (u,v) \in E \text{ or } (v,u) \in E\}$.

\begin{definition}[Congruence Relation]

Let $\mathcal{T}$ be a set of terms.
A relation $R \subseteq \mathcal{T} \times \mathcal{T}$ is a congruence relation, if has the following four properties:
\begin{itemize}
	\item reflexive: for all $t \in \mathcal{T}: (t,t) \in R$
	\item symmetric: $(s,t) \in R$ implies $(t,s) \in R$
	\item transitive: $(r,s) \in R$ and $(s,t) \in R$ implies $(r,t) \in R$
	\item compatible: $f$ is a n-ary function symbol and for all $i = 1,\ldots,n (t_i,s_i) \in R$ implies $f(t_1,\ldots,t_n),f(s_1,\ldots,s_n) \in R$
\end{itemize}

\end{definition}

Clearly every congruence relation is also an equivalence relation (which is a reflexive, transitive and symmetric relation).
Therefore every congruence relation partitions its underlying set of terms $\mathcal{T}$ into congruence classes, such that two terms $(s,t)$ belong to the same class if and only if $(s,t) \in R$.
The relations $\emptyset$ and $\mathcal{T} \times \mathcal{T}$ are trivial congruence relations.
In this work we are interested in congruence relations induced by sets of equations.
In other words, we compute the partitioning of the terms such that two terms in the same partition are proven to be equal by the input set of equations.
To this end we define the notion of congruence closure of a set of equation.

\begin{definition}[Congruence Closure]

Let $E$ be a set of equations.
The set $E^* \supseteq E$ is called the \emph{congruence closure} of $E$, 
if $E^*$ is a congruence relation on $\mathcal{T}_E$ and for every congruence relation $C$, such that $C \supset E$ follows $C \supseteq E^*$.
It is easily seen that congruence relations are closed under intersection.
Therefore $E^*$ always exists.

\noindent We write $E \models s \thickapprox t$ if $(s,t) \in E^*$ and say that $E$ is an \emph{explanation} for $s \thickapprox t$.
We call the pair $(s,t)$ an \emph{equality} and we call an equality of compound terms $f(a,b),f(c,d)$ such that $E \models a \thickapprox c$ and $E b \thickapprox d$ a \emph{deduced equality}.

\end{definition}

The idea of using congruence closure in proof compression is to replace big, w.r.t. cardinality, explanations by smaller ones.

The main theme of this work is resolution proof compression.
Therefore we need to express equality reasoning in terms of resolution proofs.
We extend the resolution calculus, presented in Section \ref{sec:resolution}, with the axioms of equality.
First we extend the notions of atoms, literals and clauses

\begin{definition}[Equality atom, literal and clause]

Let $\mathcal{T}$ be a set of terms and let $V$ be a finite set of propositional variables.
The set of \emph{equality atoms} $\mathcal{E}$ is defined as $V \cup \mathcal{T} \times \mathcal{T}$.
An \emph{equality literal} is an equality atom $e$ or a negated equality atom $\neg e$.
An \emph{equality clause} is a set of equality literals.

\end{definition}

In the context of equality atoms, we write equations $(s,t) \in \mathcal{T} \times \mathcal{T}$ as $s = t$ and $s \neq t$ for its negated version.
As usual, a clause is interpreted as the disjunction of its literals and a set of clauses is interpreted as the conjunction its clauses.\\

From hereon, we restrict our attention to sets of terms that, on top of constants, have at most one function symbol $f$, which is binary.
We justify this restriction in Section \ref{subsec:algorithms_preliminaries}.
The axioms defining congruence relations have to be reflected in our extended resolution proofs.
We achieve this by defining axiom schemas, that can be instantiated with concrete terms.

\begin{definition}[Axioms of Equality]

In the following axioms schemas, the variables $x_1,\ldots,x_n$ are placeholders for terms.
By simultaneously replacing all variables with terms, one obtains an equality clause, which we call an \emph{instance} of the respective axiom of equality.

\begin{itemize}
	\item reflexive: $\{x = x\}$
	\item symmetric: $\{x_1 \neq x_2, x_2 = x_1\}$
	\item transitive: $\{x_1 \neq x_2, x_2 \neq x_3, \ldots, x_{n-1} \neq x_n, x_1 = x_n\}$
	\item compatible: $\{x_1 \neq x_3, x_2 \neq x_4, f(x_1,x_2) = f(x_3,x_4)\}$
\end{itemize}

\end{definition}

Next we will define the resolution calculus extended by congruence axioms.

\begin{definition}[Resolution with Equality]

Let $\ell$ be an equality literal and $C_1$, $C_2$ be equality clauses such that $\ell \in C_1$ and $\neg \ell \in C_2$.
The clause $C_1 \setminus \{\ell\} \cup C_2 \setminus \{\neg \ell\}$ is the \emph{resolvent} of $C_1$ and $C_2$ with \emph{pivot} $\ell$.

\noindent Let $F = \{C_1, \ldots, C_n\}$ be a set of clauses.
The notion of a \emph{congruence derivation} for $F$ is defined inductively.
\begin{itemize}
	\item $\langle C_1, \ldots, C_n\rangle$ is a congruence derivation for $F$.
	\item If $\langle C_1, \ldots, C_m\rangle$ is a congruence derivation for $F$ then $\langle C_1, \ldots, C_{m+1} \rangle$ is a congruence derivation for $F$ if $C_{m+1}$ is an instance of an axiom of equality or $C_{m+1}$ is a resolvent of $C_i$ and $C_j$ with $1 \leq i,j \leq m$.
\end{itemize}
A \emph{congruence refutation} is a congruence derivation containing the empty clause.

\noindent Let $D = \langle C_1, \ldots, C_m\rangle$ be a congruence derivation.
The longest subsequence $\langle C_{i_1}, \ldots, C_{i_k}\rangle$ of $D$, such that $C_{i_1}, \ldots, C_{i_k}$ all are instances of axioms of equality, is called the equality reasoning part of $D$.

\end{definition}

Just like resolution derivations, congruence derivations can be visualized as directed acyclic graphs.

\begin{proposition}[Sound- \& Completeness]

Let $E$ be a set of equations and $s,t \in \mathcal{T}_E$, then $E \models s \thickapprox t$ if and only if there is a congruence refutation for $E \cup \{ s \neq t\}$

\end{proposition}

\begin{proof}

The existence of a congruence refutation in case $E \models s \thickapprox t$ is proven in terms of a proof producing algorithm, presented in Section \ref{sec:sec:proofproduction}.
This algorithm produces a congruence derivation with last clause $\{u_1 \neq v_1,\ldots,u_n \neq v_n, s = t\}$ such that $\{(u_i,v_i) \mid i = 1,\ldots,n\} \subseteq E$.
Clearly this proof can be extended to a congruence derivation for $E \cup \{ s \neq t\}$.

{\color{blue} Other implication is still open}

\end{proof}
