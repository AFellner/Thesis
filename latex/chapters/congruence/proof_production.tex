\section*{Proof Production}
\label{sec:proofproduction}

In this section we describe how to produce resolution proofs from paths in a congruence graph.
The method to carry out this operation is \texttt{prodcureProof}.
The basic idea is to traverse the path, creating a transitivity chain of equalities between adjacent nodes, while keeping track of the deduced equalities in the chain.
From invariant Deduced Edges follows that for the deduced equalities there have to be paths between the respective arguments of the compound terms.
These paths are transformed into proof recursively and resolved with a suiting instance of the congruence axiom.
Afterwards the subproof is resolved with the original transitivity chain.
Since terms can never be equal to their subterms, the procedure will eventually terminate.
The result of this procedure is a resolution proof with a root, such that the equations of the negative literals are an explanation of the target equality.
In other words, let $s \thickapprox t$ be the equality to be explained and suppose \texttt{produceProof} returns a proof with root $\rho$.
Then for $\rho$ it is the case that $F := \{(u,v) \mid u \neq v \text{ is a literal in } \rho\} \models s \thickapprox t$ and $F$ is a subset of the input equations.

\input{chapters/congruence/algorithms/prodproof}


\begin{example}

Consider again the congruence graph shown in Figure \ref{fig:short_expl} and suppose we want a proof for $a \thickapprox b$.
Suppose we found the path $p_1 := \langle  a, f(c_1,e), f(c_4,e), c_1, c_2, c_3, c_4, b \rangle$ as an explanation and that the explanation for $f(c_1,e) \thickapprox f(c_4,e)$ is the path $\langle c_1, c_2, c_3, c_4 \rangle$.
We transform $p_1$ and $p_2$ into instances of the transitivity axiom $C_1$ and $C_2$ respectively. 
The clause $C_2$ is resolved with the instance of the congruence axiom $C_3$, which is then resolved with the instance of the reflexive axiom $C_4$ resulting in clause $C_5$.
Finally, $C_1$ is resolved with $C_5$ to obtain the final clause $C_6$.
The proof is shown in Figure \ref{fig:proofprod}.

\begin{figure}[!h]
\input{chapters/congruence/figures/proof_prod}
\caption{Example proof}
\label{fig:proofprod}
\end{figure}

\end{example}

As mentioned in Section \ref{sec:congruenceclosurealgorithm}, edges are inserted into a congruence graph in a lazy way by the congruence closure algorithm.
The reason is that \texttt{produceProof} searches for explanations for edges with label $null$.
Should the equality of question be an input equation that is added later to the congruence structure than it was deduced, then we would like to overwrite this label with the input equation.
The impact of lazy insertion gets bigger, if an implementation searches for explanations already when an edge is added to the graph.
Example \ref{example:lazy} shows how this technique can help producing shorter proofs.

\begin{example}

Suppose we want to add the following sequence of equations into an empty congruence structure: $\langle (a,b),(f(a,a),d),(f(b,b),e),(f(a,a),f(b,b)) \rangle$.
After adding the first three equations, the congruence closure algorithm detects the deduced equality $f(a,a) \thickapprox f(b,b)$.
The explanation for this equality is $\{(a,b)\}$, if we were to insert the edge $(f(a,a),f(b,b))$ into the graph immediately, it would have weight 1 and label $null$.
Depending on the congruence graph used, when adding the fourth equation $(f(a,a),f(b,b))$ to the congruence structure, either the edge $(f(a,a),f(b,b))$ is not added at all to the graph or is added with weight 1.
In the latter case, both edges have weight 1 and equal chance to be selected by the shortest path algorithm.
However, choosing the edge with label $null$ is undesirable, since it two extra resolution nodes (corresponding to the compatible axiom and an intermediate node).

\end{example}

\FloatBarrier

\subsection*{Congruence Compressor}

In Section \ref{TODO} processing of a proof was defined.
The most important application of proof processing for this work is proof compression.
We want to make use of the short explanations found by the congruence closure algorithm described above.
To this end we replace subproofs with conclusions that contain unnecessary long explanations with new proofs that have shorter conclusions.
Shorter conclusions lead to less resolution steps further down the proof and possibly big chunks of the proof can simply be discarded.
There is however a tradeoff in overall proof length when introducing new subproofs.
The subproof corresponding to a short explanation can be longer in proof length, i.e. involve more resolution nodes, than one with a longer explanation.
Example \ref{example:shortexpl} displays this issue.
Additionally it can be the case that by introducing a new subproof, we only partially remove the old subproof.
Some nodes of the old subproof might still be used in other parts of the proof.
Therefore the replacement of a subproof by another, smaller one does not necessarily lead to a smaller proof.
Nevertheless, the meta heuristic favoring smaller conclusions should still dominate such effects, especially on large proofs.
The results in Section \ref{TODO} confirm this intuition.

\begin{example}
\label{example:shortexpl}
Consider the set of equations $E = \{(f(f(a,b),f(a,a)),a),(a,b),(b,f(f(b,a),f(b,b)))\}$ and the target equality $f(f(a,b),f(a,a)) \thickapprox f(f(b,a),f(b,b))$.
For presentation purposes, throughout this example we will abbreviate the term $f(f(a,b),f(a,a))$ with $t_a$ and $f(f(b,a),f(b,b))$ with $t_b$.
Using equations in $E$, one can prove the the target equality in two ways.
Either one uses the instance of the transitivity axiom $\{t_a \neq a, a \neq b, b \neq t_b, t_a = t_b\}$ or a repeated applications of instances of the congruence axiom, e.g. $\{a \neq b, f(a,a) = f(b,b)\}$.
The corresponding explanations are $E$ and $\{(a,b)\}$.

The two resulting proofs are shown in Figure \ref{fig:short_expl_proof}.
The proof with the longer explanation $E$ is only one proof node, whereas the proof with the singleton explanation has proof length 5.

%\begin{figure}[!h]
%\input{chapters/congruence/figures/shortexpl_1}
%\caption{Long explanation, short proof}
%\label{fig:short_expl_proof_1}
%\end{figure}
%
%\begin{figure}[!h]
%\input{chapters/congruence/figures/shortexpl_2}
%\caption{Short explanation, long proof}
%\label{fig:short_expl_proof_2}
%\end{figure}

\begin{figure}[!h]
%\input{chapters/congruence/figures/shortexpl_no_inpsut}
\caption{Short explanation, long proof}
\label{fig:short_expl_proof}
\end{figure}

\end{example}

The Congruence Compressor compresses processes a proof replacing subproofs as described above. 
It is defined upon the processing function specified in pseudocode in Algorithm \ref{algo:compressor}.
The idea of the processing function is simple.
Axioms are not changed by the function.
For all other nodes the \texttt{fixNode} is called method, to maintain a correct proof.
Then in line \ref{criteria} it is decided whether the explanation finding congruence closure algorithm should be used to find a replacement for the current node.
One trivial criteria could be true for every node.
Testing every node will result in a slow algorithm, but the best possible compression.
Some nodes do not need to be checked, since they contain optimal explanations by definition or there is no hope of finding an explanation at all.
The following definition classifies nodes to define a more sophisticated decision criteria.

\begin{definition}[Types of nodes]

An axiom is a \emph{theory lemma} if it is an instance of one of the congruence axioms.
Otherwise it is called \emph{input derived}.
The classification of internal nodes is defined recursively.
An internal node is input derived, if one of its premises is input derived.
Otherwise it is a theory lemma.
We call a node a \emph{low theory lemma} if it is a theory lemma and has a child that is input derived.

\end{definition}

We suspect that most redundancies in proofs are to be found in low theory lemmas, since they reflect the explanations found by the proof producing solver.
Therefore an alternative criteria is to only find replacements for low theory lemmas.
The question whether a node is a low theory lemma is not trivial to answer while traversing the proof in a top to bottom fashion.
Therefore a preliminary traversal is necessary to determine the classification of nodes.
Experiments have shown that using this criteria speeds up the algorithm a lot, while losing only very little compression.

Add all equations of the antecedent to an empty congruence structure and check whether these equations induce a proof for one of the equations in the succedent that has a shorter conclusion than the original subproof.
If there is such a proof, we replace the old subproof by the new one.
Futher criteria for deciding whether to replace or not could be size of the subproof or a global metric that tries to predict the global compression achieved by replacement.

\begin{algorithm}[h]
\caption[.]{compress}
	\KwIn{resolution node $n$}
  \KwIn{$pr:$ tuple of resolution nodes $(p_1,p_2)$ or null}
	\KwOut{resolution node}
	
	\eIf{$pr = null$}{
		\Return $n$
	}{
		$m \leftarrow fixNode(n,(p_1,p_2))$ \\
		$lE \leftarrow \{(a,b) \mid (a \neq b) \in m\}$ \\
		$rE \leftarrow \{(a,b) \mid (a = b) \in m\}$ \\
		$con \leftarrow $ empty congruence structure \\
		\For{$(a,b)$ in $lE$}{
			$con \leftarrow con.addEquality(a,b)$
		}
		\For{$(a,b)$ in $rE$}{
			$con \leftarrow con.addNode(a).addNode(b)$ \\
			$proof \leftarrow con.prodProof(s,t)$ \\
			\If{$proof \neq null$ and $\# proof.conclusion < \# m.conclusion$}{
				$m \leftarrow proof$
			}
		}
		\Return $m$
	}
  \label{algo:compressor}
\end{algorithm}


The compressor (Algorithm \ref{algo:compressor}) uses the method \texttt{fixNode} to maintain a correct proof.
The method modifies nodes with premises that have earlier been replaced by the compressor. 
Nodes with unchanged premises are not changed.
Let $n$ be a proof node that was derived using pivot $\ell$ in the original proof and which updated premises are $pr_1$ and $pr_2$ .
Depending on the presence of $\ell$ in $pr_1$ and $pr_2$, $n$ is either replaced by the resolvent of $pr_1$ and $pr_2$ or by one of the updated premises.
It assumed that the values $pr_1$, $pr_2$ and $\ell$ are stored together with the node and can be accessed in constant time.
In case both updated premises do not contain the original pivot element, replacing the node by either one of them maintains a correct proof.
Since we are interested in short proofs, we return the one with the shorter clause.
This method of maintaining a correct proof was proposed in \cite{Bar-Ilan2008} in the context of similar proof compression algorithms.

\input{chapters/congruence/algorithms/fixNode}

\FloatBarrier
