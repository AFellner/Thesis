\section*{Proof Production}
\label{sec:proofproduction}

In this section we describe how to produce resolution proofs from paths in a congruence graph.
The method to carry out this operation is \texttt{prodcureProof}.
The basic idea is to traverse the path, creating a transitivity chain of equalities between adjacent nodes, while keeping track of the deduced equalities in the chain.
From invariant Deduced Edges follows that for the deduced equalities there have to be paths between the respective arguments of the compound terms.
These paths are transformed into proof recursively and resolved with a suiting instance of the congruence axiom.
Afterwards the subproof is resolved with the original transitivity chain.
Since terms can never be equal to their subterms, the procedure will eventually terminate.
The result of this procedure is a resolution proof with a root, such that the equations of the negative literals are an explanation of the target equality.
In other words, let $s \thickapprox t$ be the equality to be explained and suppose \texttt{produceProof} returns a proof with root $\rho$.
Then for $\rho$ it is the case that $F := \{(u,v) \mid u \neq v \text{ is a literal in } \rho\} \models s \thickapprox t$ and $F$ is a subset of the input equations.

\input{chapters/congruence/algorithms/prodproof}


\begin{example}

Consider again the congruence graph shown in Figure \ref{fig:short_expl} and suppose we want a proof for $a \thickapprox b$.
Suppose we found the path $p_1 := \langle  a, f(c_1,e), f(c_4,e), c_1, c_2, c_3, c_4, b \rangle$ as an explanation and that the explanation for $f(c_1,e) \thickapprox f(c_4,e)$ is the path $\langle c_1, c_2, c_3, c_4 \rangle$.
We transform $p_1$ and $p_2$ into instances of the transitivity axiom $C_1$ and $C_2$ respectively. 
The clause $C_2$ is resolved with the instance of the congruence axiom $C_3$, which is then resolved with the instance of the reflexive axiom $C_4$ resulting in clause $C_5$.
Finally, $C_1$ is resolved with $C_5$ to obtain the final clause $C_6$.
The proof is shown in Figure \ref{fig:proofprod}.
%
%\begin{align*}
%C_1 &:= \{a \neq f(c_1,e), f(c_1,e) \neq f(c_4,e), f(c_4,e) \neq c_1, c_1 \neq c_2, c_2 \neq c_3, c_3 \neq c_4, c_4 \neq b, a = b\} \\
%C_2 &:= \{c_1 \neq c_2, c_2 \neq c_3, c_3 \neq c_4, c_1 = c_4\} \\
%C_3 &:= \{e \neq e, c_1 \neq c_4, f(c_1,e) = f(c_4,e)\} \\
%C_4 &:= \{e = e\} \\
%C_5 &:= \{c_1 \neq c_2, c_2 \neq c_3, c_3 \neq c_4, f(c_1,e) = f(c_4,e)\} \\
%C_6 &:= \{a \neq f(c_1,e), f(c_4,e) \neq c_1, c_1 \neq c_2, c_2 \neq c_3, c_3 \neq c_4, c_4 \neq b, a = b\} \\
%C_7 &:= \{a = b\} 
%\end{align*}

\begin{figure}[!h]
\input{chapters/congruence/figures/proof_prod}
\caption{Example proof}
\label{fig:proofprod}
\end{figure}


\FloatBarrier

\subsection*{Congruence Compressor}

In Section \ref{TODO} processing of a proof was defined.
The most important application of proof processing for this work is proof compression.
We want to make use of the short explanations found by the congruence closure algorithm described above.
To this end we replace subproofs with conclusions that contain unnecessary long explanations with new proofs that have shorter conclusions.
Shorter conclusions lead to less resolution steps further down the proof and possibly big chunks of the proof can simply be discarded.
There is however a tradeoff in overall proof length when introducing new subproofs.
The subproof corresponding to a short explanation can be longer in proof length, i.e. involve more resolution nodes, than one with a longer explanation.
Example \ref{example:shortexpl} displays this issue.
Additionally it can be the case that by introducing a new subproof, we only partially remove the old subproof.
Some nodes of the old subproof might still be used in other parts of the proof.
Therefore the replacement of a subproof by another, smaller one does not necessarily lead to a smaller proof.
Nevertheless, the meta heuristic favoring smaller conclusions should still dominate such effects, especially on large proofs.
The results in Section \ref{TODO} confirm this intuition.

\begin{example}
\label{example:shortexpl}
Consider the set of equations $E = \{(f(f(a,b),f(a,a)),a),(a,b),(b,f(f(b,a),f(b,b)))\}$ and the target equality $f(f(a,b),f(a,a)) \thickapprox f(f(b,a),f(b,b))$.
For presentation purposes, throughout this example we will abbreviate the term $f(f(a,b),f(a,a))$ with $t_a$ and $f(f(b,a),f(b,b))$ with $t_b$.
Using equations in $E$, one can prove the the target equality in two ways.
Either one uses the instance of the transitivity axiom $\{t_a \neq a, a \neq b, b \neq t_b, t_a = t_b\}$ or a repeated applications of instances of the congruence axiom, e.g. $\{a \neq b, f(a,a) = f(b,b)\}$.
The corresponding explanations are $E$ and $\{(a,b)\}$.

The two resulting proofs are shown in Figure \ref{fig:short_expl_proof}.
The proof with the longer explanation $E$ is only one proof node, whereas the proof with the singleton explanation has proof length 5.

%\begin{figure}[!h]
%\input{chapters/congruence/figures/shortexpl_1}
%\caption{Long explanation, short proof}
%\label{fig:short_expl_proof_1}
%\end{figure}
%
%\begin{figure}[!h]
%\input{chapters/congruence/figures/shortexpl_2}
%\caption{Short explanation, long proof}
%\label{fig:short_expl_proof_2}
%\end{figure}

\begin{figure}[!h]
\input{chapters/congruence/figures/shortexpl_no_input}
\caption{Short explanation, long proof}
\label{fig:short_expl_proof}
\end{figure}

\end{example}



The Congruence Compressor compresses processes a proof replacing subproofs as described above. It is defined upon the following processing function, specified in pseudocode.
Input equations are assumed to be true, i.e. an input equation $(u,v)$ is treated as an axiomatic clause $\{u = v\}$.
In a last step of \texttt{prodProof} all input equations are resolved away.

\begin{algorithm}[h]
\caption[.]{compress}
	\KwIn{resolution node $n$}
  \KwIn{$pr:$ tuple of resolution nodes $(p_1,p_2)$ or null}
	\KwOut{resolution node}
	
	\eIf{$pr = null$}{
		\Return $n$
	}{
		$m \leftarrow fixNode(n,(p_1,p_2))$ \\
		$lE \leftarrow \{(a,b) \mid (a \neq b) \in m\}$ \\
		$rE \leftarrow \{(a,b) \mid (a = b) \in m\}$ \\
		$con \leftarrow $ empty congruence structure \\
		\For{$(a,b)$ in $lE$}{
			$con \leftarrow con.addEquality(a,b)$
		}
		\For{$(a,b)$ in $rE$}{
			$con \leftarrow con.addNode(a).addNode(b)$ \\
			$proof \leftarrow con.prodProof(s,t)$ \\
			\If{$proof \neq null$ and $\# proof.conclusion < \# m.conclusion$}{
				$m \leftarrow proof$
			}
		}
		\Return $m$
	}
  \label{algo:compressor}
\end{algorithm}


The compressor (Algorithm \ref{algo:compressor}) uses the method \texttt{fixNode} to maintain a correct proof.
The method modifies nodes with premises that have earlier been replaced by the compressor. 
Nodes with unchanged premises are not changed.
Let $n$ be a proof node that was derived using pivot $\ell$ in the original proof and which updated premises are $pr_1$ and $pr_2$ .
Depending on the presence of $\ell$ in $pr_1$ and $pr_2$, $n$ is either replaced by the resolvent of $pr_1$ and $pr_2$ or by one of the updated premises.
It assumed that the values $pr_1$, $pr_2$ and $\ell$ are stored together with the node and can be accessed in constant time.
In case both updated premises do not contain the original pivot element, replacing the node by either one of them maintains a correct proof.
Since we are interested in short proofs, we return the one with the shorter clause.
This method of maintaining a correct proof was proposed in \cite{Bar-Ilan2008} in the context of similar proof compression algorithms.

\input{chapters/congruence/algorithms/fixNode}

\FloatBarrier
