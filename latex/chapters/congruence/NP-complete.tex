\section*{NP-completeness of shortest path decision problem}

In this section we will assume that every propositional logic formula is given in conjunctive normal form and tuples of terms $(u,v)$ are understood as equations $u = v$.

\begin{definition}[Short path decision problem]

Let $E$ be a set of input equations $s_1 = t_1,\ldots,s_n = t_n$ of terms $\mathcal{T}$.
For $k \in \mathbb{N}$ and a target equation $s = t$, the short path decision problem is the question whether there exists an $E' \subseteq E$ such that $E' \models s = t$ and $|E'| \leq k$.

\end{definition}

Note: $E \models s = t$ means $s$ and $t$ are in the same congruence class of the congruence closure of $E$.

\begin{definition}[Congruence translation]

Let $\Phi$ be a propositional logic formula with clauses $C_1,\ldots,C_n$ using variables $x_1,\ldots,x_m$.
%The set of terms $\mathcal{T}$ is constructed using the following constants and function symbols.
%For every $i= 1,\ldots,n + 1$, there is a constant $\hat{c}_i$ and function symbols $t_i$ and $f_i$.
%For every $j= 1,\ldots,m$, there are constants $\hat{x}_j$, $\top_j$ and $\bot_j$.
The congruence translation $E_{\Phi}$ of $\Phi$ is defined as $E_1 \cup E_2 \cup E_3 \cup E_4$, where
\begin{align*}
	&E_1 = \{ (\hat{x}_j,\top_j), (\hat{x}_j,\bot_j) \mid 1 \leq j \leq m\} \\
	&E_2 = \{ (\hat{c}_i, t_i(\hat{x}_j)) \mid x_j \text{ appears positively in } C_i\} \\
	&E_3 = \{ (\hat{c}_i, f_i(\hat{x}_j)) \mid x_j \text{ appears negatively in } C_i\} \\
	&E_4 = \{ (t_i(\top_j),\hat{c}_{i+1}),(f_i(\bot_j), \hat{c}_{i+1}) \mid 1 \leq i \leq n+m, 1 \leq j \leq m\} \\
%\end{align*}
&\hspace{-1cm}\text{Additionally, for every $i = 1,\ldots, n$ and $j = 1,\ldots, m$ we define the following sets }\\
%\begin{align*}
&T_{ij} = \{(\hat{c}_i, t_i(\hat{x}_j)), (\hat{x}_j,\top_j), (t_i(\top_j),\hat{c}_{i+1})\} \\
&F_{ij} = \{(\hat{c}_i, f_i(\hat{x}_j)), (\hat{x}_j,\bot_j), (f_i(\bot_j),\hat{c}_{i+1})\}
\end{align*}

\end{definition}

\begin{lemma}[Characterization of explanations]
\label{lemma:charexpl}

Let $\Phi$ be a propositional logic formula with $n$ clauses and $m$ variables.
For every subset $E$ of $E_{\Phi}$, $E \models \hat{c}_1 = \hat{c}_{n+1}$ if and only if for every $i = 1,\ldots,n$ there is a $j = 1,\ldots,m$ such that $T_{ij} \subseteq E$ or $F_{ij} \subseteq E$.

\end{lemma}

\begin{proof}

Suppose that for every $i = 1,\ldots,n$ there is a $j = 1,\ldots,m$ such that $T_{ij} \subseteq E$ or $F_{ij} \subseteq E$.
Clearly $T_{ij} \models \hat{c}_i = t_i(\hat{x}_j)$ and $T_{ij} \models t_i(\top_j) = \hat{c}_{i+1}$. 
Since $(\hat{x}_j,\top_j) \in E$ the fact $E \models t_i(\hat{x}_j) = t_i(\top_j)$ follows by an application of the deduction axiom.
Using the transitivity axiom it follows that $T_{ij} \models \hat{c}_i = \hat{c}_{i+1}$.
Similarly it can be shown that $F_{ij} \models \hat{c}_i = \hat{c}_{i+1}$.
Therefore it follows from the assumption that $E \models \hat{c}_i = \hat{c}_{i+1}$ for every $i = 1,\ldots,n$.
Using the transitivity axiom it follows that $E \models \hat{c}_1 = \hat{c}_{n+1}$.


\noindent We will show the other direction of the equivalence by induction on $n$.
\begin{paragraph}{Induction Base $n = 1$:}

Suppose that $E \models \hat{c}_1 = \hat{c}_{2}$. %, which implies $\hat{c}_2 \in [\hat{c}_1]_E$.
Since $\hat{c}_1$ is a constant, the deduction axiom can not be applied to it.
Therefore in order to satisfy $E \models \hat{c}_1 = t$ with $t \neq \hat{c}_1$ there has to be an equation $(\hat{c}_1, t) \in E$ for some term $t$.
Since $E \subseteq E_{\Phi}$, the only possible such equations are of the form $(\hat{c}_1, t_1(\hat{x}_j))$ and $(\hat{c}_1, f_1(\hat{x}_j))$ for some $j$.
The only equations in $E$ with involving terms with the function symbols $t_1$ and $f_1$ are of the form $(\hat{c}_1, t_1(\hat{x}_j)), (t_1(\top_j),\hat{c}_2)$ and $(\hat{c}_1, f_1(\hat{x}_j)), (f_1(\bot_j),\hat{c}_2)$.
Therefore in order to satisfy $E \models \hat{c}_1 = t$ such that $t$ is neither the constant $\hat{c}_1$ nor some term $t_1(\hat{x}_j), f_1(\hat{x}_j)$, it is necessary that $E \models t_1(\hat{x}_j) = t_1(\top_j)$ and $(\hat{c}_1, t_1(\hat{x}_j)) \in E$ or $E \models f_1(\hat{x}_j) = f_1(\bot_j)$ and $(f_1(\bot_j),\hat{c}_2) \in E$ for some $j$.
The conditions can only be satisfied with equations of $E_{\Phi}$ if $\{(\hat{c}_1, t_1(\hat{x}_j)), (\hat{x}_j,\top_j)\} \subseteq E$ or $\{(\hat{c}_1, f_1(\hat{x}_j), (\hat{x}_j,\bot_j)\} \subseteq E$ respectively.
%Similarly $t \in [\hat{c}_1]_E$ such that $t$ is neither $\hat{c}_1$ nor a term with the function symbol $t_1$ or $f_1$, for some $j$ $(t_1(\top_j),\hat{c}_2) \in E$ and 
From a similar argumentation about the equations involving $c_2$ and $t_1(\top_j)$ or $f_1(\bot_j)$ it follows that either $T_{1j} \subseteq E$ or $F_{1j} \subseteq E$ for some $j$.
\end{paragraph}

\begin{paragraph}{Induction Hypothesis:} For every subset $E$ of $E_{\Phi}$, $E \models \hat{c}_1 = \hat{c}_{n}$ if and only if for every $i = 1,\ldots,n-1$ there is a $j = 1,\ldots,m$ such that $T_{ij} \subseteq E$ or $F_{ij} \subseteq E$.
\end{paragraph}

\begin{paragraph}{Induction Step:}
Suppose that $E \models \hat{c}_1 = \hat{c}_{n+1}$.\\
Similarly to the argumentation in the induction base, the only equations in $E_{\Phi}$ involving $\hat{c}_{n+1}$ are of the form $(t_n(\top_j),\hat{c}_{n+1})$ and $(f_n(\bot_j),\hat{c}_{n+1})$.
The only way to enrich the congruence class of $\hat{c}_{n+1}$ beyond $\hat{c}_{n+1}$ and terms of the form $t_n(\top_j)$ and $f_n(\bot_j)$ 
is that for some $j$, $(\hat{x}_j,\top_j) \in E$ or $(\hat{x}_j,\bot_j) \in E$ and subsequently also $(\hat{c}_n,t_n(\hat{x}_j)) \in E$ or $(\hat{c}_n,f_n(\hat{x}_j)) \in E$.
Thus $T_{nj} \subseteq E$ or $F_{nj} \subseteq E$ and as a consequence $E \models \hat{c}_n = \hat{c}_{n+1}$.
Using transitivity $E \models \hat{c}_1 = \hat{c}_{n+1}$ and $E \models \hat{c}_n = \hat{c}_{n+1}$ imply $E \models \hat{c}_1 = \hat{c}_n$ and from the induction hypothesis it follows that for every $i = 1,\ldots,n-1$ $T_{ij} \subseteq E$ or $F_{ij} \subseteq E$.
\end{paragraph}

\end{proof}

\begin{lemma}[NP - hardness]

The short path decision problem is NP - hard.

\end{lemma}

\begin{proof}

We will reduce SAT to the short path decision problem.
Let $\Phi$ be a propositional formula in conjunctive normal form with variables $x_1,\ldots,x_m$ and clauses $C_1,\ldots,C_n$.
Let $C_{n+1},\ldots,C_{n+m}$ be topological clauses $\{x_1, \neg x_1\},\ldots,\{x_m,\neg x_m\}$.
Clearly $\Phi$ is satisfiable if and only if $\Phi' = \{c_1,\ldots,c_{n+m}\}$ is satisfiable.
We will show that $\Phi'$ is satisfiable if and only if there exists $E \subseteq E_{\Phi'}$ such that $E \models \hat{c}_1 = \hat{c}_{n+m+1}$ and $|E| \leq 2n + 3m$.

\emph{Suppose} $\Phi'$ is satisfiable and let $\mathcal{I}$ be a satisfying assignment.\\
For every clause $C_i$ there is a literal $\ell_i \in C_i$ such that $\mathcal{I} \models \ell_i$.
For every $i = 1,\ldots,n+m$ we define $E_i$ to be $T_{ij}$ if $\ell_i = x_j$ and $F_{ij}$ if $\ell_i = \neg x_j$.
From $\ell_i \in C_i$ it follows $E_i \subseteq E_{\Phi'}$.
Let $E = \bigcup_i^n E_i$ then from Lemma \ref{lemma:charexpl} follows $E \models c_1 = c_{n+m+1}$.
What remains to show is that $|E| \leq 2n + 3m$.
Since the sets $E_2, E_3$ and $E_4$ in the definition of $E_{\Phi'}$ are pairwise disjoint, for $i \neq j$ $E_i \cap E_j \subseteq \{(\hat{x}_j,\top_j),(\hat{x}_j,\bot_j) \mid j = 1,\ldots,m\}$.
Therefore $E$ involves exactly $2(n+m)$ equations of $E_2, E_3$ and $E_4$.
By construction of the sets $E_i$ and the clauses $C_{n+1},\ldots,C_{n+m}$ there is no $j = 1,\ldots,m$ such that $(\hat{x}_j,\top_j) \in E$ and $(\hat{x}_j,\bot_j) \in E$.
Therefore $E$ involves $m$ equations of set $E_1$ in the definition of $E_{\Phi'}$.
Overall we have $|E| = 2n + 3m$.

\emph{Suppose} there exists $E \subseteq E_{\Phi'}$, $E \models \hat{c}_1 = \hat{c}_{n+m+1}$ and $|E| \leq 2n + 3m$.\\
We will show that $\mathcal{I} = \{\hat{x}_j \mid (\hat{x}_j,\top_j) \in E\}$ is a satisfying assignment for $\Phi'$ represented as a set of true variables.
Let $C_i$ be an arbitrary clause of $\Phi'$.
From $E \models \hat{c}_1 = \hat{c}_{n+m+1}$ using Lemma \ref{lemma:charexpl} follows $T_{ij} \subseteq E$ or $F_{ij} \subseteq E$.
Assume $T_{ij} \subseteq E$ for some $j = 1,\ldots,m$.
$E \subseteq E_{\Phi'}$ implies that $x_j$ appears positively in $C_i$. 
By definition of $\mathcal{I} \models x_j$. 
Therefore $\mathcal{I} \models C_i$.
Similarly $\mathcal{I} \models C_i$ in case $F_{ij} \subseteq E$ for some $j = 1,\ldots,m$.
Since $i$ was arbitrary, $\mathcal{I} \models \Phi'$.

\end{proof}