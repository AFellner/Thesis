\section{NP-completeness of Short Explanation Decision Problem}
\label{sec:npcomplete}
In Section \ref{sec:congruencedef} the notion of an explanation is defined and it was mentioned that we want to find short explanations in order to compress proofs.
In this section we show that one might have to search a while to find the shortest one, by proving that the problem of deciding whether there is an explanation of a given size is NP-complete.
Our proof of NP-completeness reduces the problem of deciding the satisfiability of a propositional logic formula  in conjunctive normal form (SAT) to the short explanation decision problem.
For basics about satisfiability of propositional logic formulas and assignments, we refer the reader to \cite{Biere2009}.
We begin by formally defining the problem.

\begin{definition}[Short explanation decision problem]

Let $E = \{(s_1,t_1),\ldots,(s_n,t_n)\}$ be a set of equations, $k \in \mathbb{N}$ and $(s,t)$ be a target equation.
The \emph{short explanation decision problem} is the question whether there exists a set $E'$ such that $E' \subseteq E$, $E' \models s \thickapprox t$ and $|E'| \leq k$.

\end{definition}

Our proof of hardness translates propositional formulas into sets of equations and proves properties of the resulting formulas.
To this end we define a general translation of formulas in conjunctive normal form into sets of equations.

\begin{definition}[Congruence Translation]

Let $\Phi$ be a propositional logic formula in conjunctive normal form with clauses $C_1,\ldots,C_n$ using variables $x_1,\ldots,x_m$.
%The set of terms $\mathcal{T}$ is constructed using the following constants and function symbols.
%For every $i= 1,\ldots,n + 1$, there is a constant $\hat{c}_i$ and function symbols $t_i$ and $f_i$.
%For every $j= 1,\ldots,m$, there are constants $\hat{x}_j$, $\top_j$ and $\bot_j$.
The congruence translation $E_{\Phi}$ of $\Phi$ is defined as the set of equations $Assignment \cup Pos \cup Neg \cup Connect$, where
\begin{align*}
	Assignment &= \{ (\hat{x}_j,\top_j), (\hat{x}_j,\bot_j) \mid 1 \leq j \leq m\} \\
	Pos &= \{ (\hat{c}_i, t_i(\hat{x}_j)) \mid x_j \text{ appears positively in } C_i\} \\
	Neg &= \{ (\hat{c}_i, f_i(\hat{x}_j)) \mid x_j \text{ appears negatively in } C_i\} \\
	Connect &= \{ (t_i(\top_j),\hat{c}_{i+1}),(f_i(\bot_j), \hat{c}_{i+1}) \mid 1 \leq i \leq n, 1 \leq j \leq m\} \\
%\end{align*}
&\hspace{-2cm}\text{For presentation purposes we define the following sets for every $i = 1,\ldots, n$ and $j = 1,\ldots, m$}\\
%\begin{align*}
T_{ij} &= \{(\hat{c}_i, t_i(\hat{x}_j)), (\hat{x}_j,\top_j), (t_i(\top_j),\hat{c}_{i+1})\} \\
F_{ij} &= \{(\hat{c}_i, f_i(\hat{x}_j)), (\hat{x}_j,\bot_j), (f_i(\bot_j),\hat{c}_{i+1})\}
\end{align*}

\end{definition}

The following examples show the congruence translation of a propositional formula and a subset of the translation corresponding to a satisfying assignment.
We use the standard notion of satisfiability and present variable assignments as sets of those propositional variables being mapped to true.

\begin{example}

Let $\Phi := (x_1 \vee x_2 \vee \neg x_3) \wedge (\neg x_2 \vee x_3) \wedge (\neg x_1 \vee \neg x_2)$. 
Figures \ref{fig:npexamplebig}, \ref{fig:npassignment} and \ref{fig:npexplanation} present sets of equations graphically, where an edge between two nodes means that the respective set contains an equation between the two nodes.
Figure \ref{fig:npexamplebig} shows the graphical representation of the equations in $Pos, Neg$ and $Connect$ for the congruence translation $E_{\Phi}$ of $\Phi$.
Let $\mathcal{I} := \{x_1,x_3\}$. 
It is easy to see that $\mathcal{I} \models \Phi$.
Figure \ref{fig:npexplanation} shows a graphical representation of $\mathcal{I}$ in terms of equations.
This set of equations is an explanation for $(\hat{c}_1,\hat{c}_4)$.
Note that $\mathcal{I}' := \{x_1\}$ is another satisfying assignment and for the satisfiability of $\Phi$, the truth value of variable $x_3$ is not essential.
Therefore replacing $\{(\hat{x}_3,\top_3)\}$ by $\{(\hat{x}_2,\top_2)\}$ in the explanation corresponding to $\mathcal{I}$ leads to another explanation of $(\hat{c}_1,\hat{c}_4)$ of equal size.
However, this set does not uniquely represent an assignment, since it is not clear which truth value $x_2$ has.
In the proof of Lemma \ref{lemma:nphardness} we exclude such ambiguous sets by introducing additional topological clauses.
\begin{figure}[!h]


\centering
\begin{tikzpicture}[node distance=.3cm]

\node(c1){$\hat{c}_1$};

\node[right =.4cm of c1] (t1x2) {$t_1(\hat{x}_2)$};
\node[above =1cm of t1x2] (t1x1) {$t_1(\hat{x}_1)$};
\node[below =1cm of t1x2] (f1x3) {$f_1(\hat{x}_3)$};

\draw [-] (c1) to (t1x2);
\draw [-] (c1) to (t1x1);
\draw [-] (c1) to (f1x3);


\node[right =.4cm of t1x1, yshift = 0.25cm] (f11) {$f_1(\bot_1)$};
\node[above =.4cm of f11] (t11) {$t_1(\top_1)$};
\node[below =.4cm of f11] (t12) {$t_1(\top_2)$};
\node[below =.4cm of t12] (f12) {$f_1(\bot_2)$};
\node[below =.4cm of f12] (t13) {$t_1(\top_3)$};
\node[below =.4cm of t13] (f13) {$f_1(\bot_3)$};

\node[right = 4cm of c1](c2){$\hat{c}_2$};

\draw [-] (c2) to (t11);
\draw [-] (c2) to (t12);
\draw [-] (c2) to (t13);
\draw [-] (c2) to (f11);
\draw [-] (c2) to (f12);
\draw [-] (c2) to (f13);

\node[right =.25cm of c2, yshift=1cm]  (f2x2) {$f_2(\hat{x}_2)$};
\node[below =1.5cm of f2x2] (t2x3) {$t_2(\hat{x}_3)$};

\draw [-] (c2) to (f2x2);
\draw [-] (c2) to (t2x3);

\node[right =3cm of f11] (f21) {$f_2(\bot_1)$};
\node[right =3cm of t11] (t21) {$t_2(\top_1)$};
\node[right =3cm of t12] (t22) {$t_2(\top_2)$};
\node[right =3cm of f12] (f22) {$f_2(\bot_2)$};
\node[right =3cm of t13] (t23) {$t_2(\top_3)$};
\node[right =3cm of f13] (f23) {$f_2(\bot_3)$};

\node[right = 4cm of c2](c3){$\hat{c}_3$};

\draw [-] (c3) to (t21);
\draw [-] (c3) to (t22);
\draw [-] (c3) to (t23);
\draw [-] (c3) to (f21);
\draw [-] (c3) to (f22);
\draw [-] (c3) to (f23);

\node[right =.25cm of c3, yshift=1cm] (f3x1) {$f_3(\hat{x}_1)$};
\node[below =1.5cm of f3x1] (f3x2) {$f_3(\hat{x}_2)$};

\draw [-] (c3) to (f3x1);
\draw [-] (c3) to (f3x2);

\node[right =3.4cm of f21] (f31) {$f_3(\bot_1)$};
\node[right =3.4cm of t21] (t31) {$t_3(\top_1)$};
\node[right =3.4cm of t22] (t32) {$t_3(\top_2)$};
\node[right =3.4cm of f22] (f32) {$f_3(\bot_2)$};
\node[right =3.4cm of t23] (t33) {$t_3(\top_3)$};
\node[right =3.4cm of f23] (f33) {$f_3(\bot_3)$};

\node[right = 4.5cm of c3](c4){$\hat{c}_4$};

\draw [-] (c4) to (t31);
\draw [-] (c4) to (t32);
\draw [-] (c4) to (t33);
\draw [-] (c4) to (f31);
\draw [-] (c4) to (f32);
\draw [-] (c4) to (f33);

\end{tikzpicture}


\caption{Pos, Neg and Connect for $E_{\Phi}$}
\label{fig:npexamplebig}
\end{figure}


\begin{figure}[!h]


\centering
\begin{tikzpicture}[node distance=.75cm]

\node (x1) {$\hat{x}_1$};
\node [below =.5cm of x1] (x2) {$\hat{x}_2$};
\node [below =.5cm of x2] (x3) {$\hat{x}_3$};

\node [right = of x1] (t1) {$\top_1$};
\node [left = of x1] (f1) {$\bot_1$};

\draw [-] (x1) to (t1);
\draw [-] (x1) to (f1);

\node [right = of x2] (t2) {$\top_2$};
\node [left = of x2] (f2) {$\bot_2$};

\draw [-] (x2) to (t2);
\draw [-] (x2) to (f2);

\node [right = of x3] (t3) {$\top_3$};
\node [left = of x3] (f3) {$\bot_3$};

\draw [-] (x3) to (t3);
\draw [-] (x3) to (f3);

\end{tikzpicture}



\caption{Assignment for $E_{\Phi}$}
\label{fig:npassignment}
\end{figure}

\begin{figure}[!h]


\centering
\begin{tikzpicture}
\tikzstyle{every node}=[font=\small]
\node(c1){$\hat{c}_1$};

\node[right =.2cm of c1] (t1x1) {$t_1(\hat{x}_1)$};

\draw [-] (c1) to (t1x1);

\node[right =.1cm of t1x1] (t11) {$t_1(\top_1)$};

%\draw [dashed] (t1x1) to (t11);

\node[right =.2cm of t11](c2){$\hat{c}_2$};

\draw [-] (c2) to (t11);

\node[right =.2cm of c2]  (f2x2) {$f_2(\hat{x}_2)$};

\draw [-] (c2) to (f2x2);

\node[right =.1cm of f2x2] (f22) {$f_2(\bot_2)$};

%\draw [dashed] (f2x2) to (f22);

\node[right = .2cm of f22](c3){$\hat{c}_3$};

\draw [-] (c3) to (f22);

\node[right =.2cm of c3] (f3x2) {$f_3(\hat{x}_2)$};

\draw [-] (c3) to (f3x2);

\node[right =.1cm of f3x2] (f32) {$f_3(\bot_2)$};

%\draw [dashed] (f3x2) to (f32);

\node[right =.2cm of f32](c4){$\hat{c}_4$};

\draw [-] (c4) to (f32);

\node [below =.5cm of t1x1] (x1) {$\hat{x}_1$};

\node [right =.5cm of x1] (t1) {$\top_1$};

\node [right =1cm of t1] (f2) {$\bot_2$};


\node [right =.5cm of f2] (x2) {$\hat{x}_2$};

\node [right =1cm of x2] (x3) {$\hat{x}_3$};


\node [right =1cm of x3] (t3) {$\top_3$};

\draw [-] (x1) to (t1);

\draw [-] (x2) to (f2);


\draw [-] (x3) to (t3);

\end{tikzpicture}


\caption{Explanation of $(\hat{c}_1,\hat{c}_4)$}
\label{fig:npexplanation}
\end{figure}

\end{example}


\begin{lemma}[Characterization of explanations]
\label{lemma:charexpl}

Let $\Phi$ be a propositional logic formula in conjunctive normal form with $n$ clauses and $m$ variables.
For every subset $E$ of $E_{\Phi}$, $E \models \hat{c}_1 \thickapprox \hat{c}_{n+1}$ if and only if for every $i = 1,\ldots,n$ there is a $j = 1,\ldots,m$ such that $T_{ij} \subseteq E$ or $F_{ij} \subseteq E$.

\end{lemma}

\begin{proof}

Suppose that for every $i = 1,\ldots,n$ there is a $j = 1,\ldots,m$ such that $T_{ij} \subseteq E$ or $F_{ij} \subseteq E$.
Clearly $T_{ij} \models \hat{c}_i \thickapprox t_i(\hat{x}_j)$ and $T_{ij} \models t_i(\top_j) \thickapprox \hat{c}_{i+1}$. 
Since $(\hat{x}_j,\top_j) \in E$, the fact $E \models t_i(\hat{x}_j) \thickapprox t_i(\top_j)$ follows by an application of the compatibility axiom.
Using the transitivity of congruence relations, it follows that $T_{ij} \models \hat{c}_i \thickapprox \hat{c}_{i+1}$.
Similarly it can be shown that $F_{ij} \models \hat{c}_i \thickapprox \hat{c}_{i+1}$.
Therefore it follows from the assumption that $E \models \hat{c}_i \thickapprox \hat{c}_{i+1}$ for every $i = 1,\ldots,n$.
Using transitivity again, it follows that $E \models \hat{c}_1 \thickapprox \hat{c}_{n+1}$.

\noindent We show the other implication of the equivalence by induction on $n$.
\begin{paragraph}{Induction Base $n = 1$:}

Suppose that $E \models \hat{c}_1 \thickapprox \hat{c}_{2}$. %, which implies $\hat{c}_2 \in [\hat{c}_1]_E$.
Since $\hat{c}_1$ is a constant, the compatibility axiom can not be applied to extend the congruence class of $\hat{c}_1$ beyond the singleton $\{\hat{c}_1\}$.
Therefore in order to satisfy $E \models \hat{c}_1 \thickapprox u$ with $u \neq \hat{c}_1$ there has to be an equation $(\hat{c}_1, u) \in E$ for some term $u$.
Since $E \subseteq E_{\Phi}$, the only possible such equations are $(\hat{c}_1, t_1(\hat{x}_j))$ and $(\hat{c}_1, f_1(\hat{x}_j))$ for some $j$.
The only equations in $E$ involving terms with the function symbols $t_1$ and $f_1$ are $(\hat{c}_1, t_1(\hat{x}_j)), (t_1(\top_j),\hat{c}_2)$ and $(\hat{c}_1, f_1(\hat{x}_j)), (f_1(\bot_j),\hat{c}_2)$ for some $j$.
Therefore in order to satisfy $E \models \hat{c}_1 \thickapprox u$ such that $u$ is neither the constant $\hat{c}_1$, nor some term $t_1(\hat{x}_j), f_1(\hat{x}_j)$, it is necessary that $E \models t_1(\hat{x}_j) \thickapprox t_1(\top_j)$ and $(\hat{c}_1, t_1(\hat{x}_j)) \in E$ or $E \models f_1(\hat{x}_j) \thickapprox f_1(\bot_j)$ and $(f_1(\bot_j),\hat{c}_2) \in E$ for some $j$.
The conditions can only be satisfied with equations of $E_{\Phi}$ if $\{(\hat{c}_1, t_1(\hat{x}_j)), (\hat{x}_j,\top_j)\} \subseteq E$ or $\{(\hat{c}_1, f_1(\hat{x}_j), (\hat{x}_j,\bot_j)\} \subseteq E$ respectively.
%Similarly $t \in [\hat{c}_1]_E$ such that $t$ is neither $\hat{c}_1$ nor a term with the function symbol $t_1$ or $f_1$, for some $j$ $(t_1(\top_j),\hat{c}_2) \in E$ and 
From a similar argumentation about the equations involving $\hat{c_2}$ and $t_1(\top_j)$ or $f_1(\bot_j)$ it follows that either $T_{1j} \subseteq E$ or $F_{1j} \subseteq E$ for some $j$.
\end{paragraph}

\begin{paragraph}{Induction Hypothesis:} For every subset $E$ of $E_{\Phi}$, $E \models \hat{c}_1 \thickapprox \hat{c}_{n}$ if and only if for every $i = 1,\ldots,n-1$ there is a $j = 1,\ldots,m$ such that $T_{ij} \subseteq E$ or $F_{ij} \subseteq E$.
\end{paragraph}

\begin{paragraph}{Induction Step:}
Suppose that $E \models \hat{c}_1 \thickapprox \hat{c}_{n+1}$.\\
Similarly to the argumentation in the induction base, the only equations in $E_{\Phi}$ involving $\hat{c}_{n+1}$ are of the form $(t_n(\top_j),\hat{c}_{n+1})$ and $(f_n(\bot_j),\hat{c}_{n+1})$.
The only possibility to enrich the congruence class of $\hat{c}_{n+1}$ with terms other than $\hat{c}_{n+1}$ and those of the form $t_n(\top_j)$ and $f_n(\bot_j)$, 
is that for some $j$, $(\hat{x}_j,\top_j) \in E$ or $(\hat{x}_j,\bot_j) \in E$ and subsequently also $(\hat{c}_n,t_n(\hat{x}_j)) \in E$ or $(\hat{c}_n,f_n(\hat{x}_j)) \in E$.
Thus $T_{nj} \subseteq E$ or $F_{nj} \subseteq E$ and as a consequence $E \models \hat{c}_n \thickapprox \hat{c}_{n+1}$.
Using transitivity $E \models \hat{c}_1 \thickapprox \hat{c}_{n+1}$ and $E \models \hat{c}_n \thickapprox \hat{c}_{n+1}$ imply $E \models \hat{c}_1 \thickapprox \hat{c}_n$ and from the induction hypothesis it follows that  $T_{ij} \subseteq E$ or $F_{ij} \subseteq E$ for every $i = 1,\ldots,n-1$.
\end{paragraph}

\end{proof}

\begin{lemma}[NP- hardness]
\label{lemma:nphardness}
The short explanation decision problem is NP- hard.

\end{lemma}

\begin{proof}

We reduce SAT to the short explanation decision problem.
SAT is a well known NP-complete problem \cite{Biere2009}.
Let $\Phi$ be a propositional formula in conjunctive normal form with clauses $C_1,\ldots,C_n$ and variables $x_1,\ldots,x_m$.
Let $C_{n+1},\ldots,C_{n+m}$ be the tautological clauses $\{x_1, \neg x_1\},\ldots,\{x_m,\neg x_m\}$.
Clearly $\Phi$ is satisfiable if and only if $\Phi' = \{C_1,\ldots,C_{n+m}\}$ is satisfiable.
We will show that $\Phi'$ is satisfiable if and only if there exists $E \subseteq E_{\Phi'}$ such that $E \models \hat{c}_1 \thickapprox \hat{c}_{n+m+1}$ and $|E| \leq 2n + 3m$.

\emph{Suppose} $\Phi'$ is satisfiable and let $\mathcal{I}$ be a satisfying assignment.\\
For every clause $C_i$ there is a literal $\ell_i \in C_i$, such that $\mathcal{I} \models \ell_i$.
For every $i = 1,\ldots,n+m$ we set $E_i:= T_{ij}$ if $\ell_i = x_j$ and $E_i:= F_{ij}$ if $\ell_i = \neg x_j$.
From $\ell_i \in C_i$ follows $E_i \subseteq E_{\Phi'}$.
Let $E = \bigcup_i^n E_i$ then from Lemma \ref{lemma:charexpl} and the transitivity of the congruence relations follows $E \models \hat{c_1} \thickapprox \hat{c_{n+m+1}}$.
What remains to show is that $|E| \leq 2n + 3m$.
Since the sets $Pos, Neg$ and $Connect$ in the definition of $E_{\Phi'}$ are pairwise disjoint, for $i \neq j$ $E_i \cap E_j \subseteq \{(\hat{x}_j,\top_j),(\hat{x}_j,\bot_j) \mid j = 1,\ldots,m\}$.
Therefore $E$ involves exactly $2(n+m)$ equations of $Pos, Neg$ and $Connect$.
By construction of the sets $E_i$ and the clauses $C_{n+1},\ldots,C_{n+m}$ there is no $j = 1,\ldots,m$ such that $(\hat{x}_j,\top_j) \in E$ and $(\hat{x}_j,\bot_j) \in E$.
Therefore $E$ involves $m$ equations of set $Assignment$ in the definition of $E_{\Phi'}$.
Overall we have $|E| = 2n + 3m$.

\emph{Suppose} there exists $E \subseteq E_{\Phi'}$, $E \models \hat{c}_1 \thickapprox \hat{c}_{n+m+1}$ and $|E| \leq 2n + 3m$.\\
We will show that $\mathcal{I} = \{\hat{x}_j \mid (\hat{x}_j,\top_j) \in E\}$ is a satisfying assignment for $\Phi'$.
Let $i = 1,\ldots,n+m$ be an arbitrary clause index.
From $E \models \hat{c}_1 \thickapprox \hat{c}_{n+m+1}$ and Lemma \ref{lemma:charexpl} follows $T_{ij} \subseteq E$ or $F_{ij} \subseteq E$ for some $j=1,\ldots,m$.
Assume $T_{ij} \subseteq E$ for some $j = 1,\ldots,m$.
$E \subseteq E_{\Phi'}$ implies that $x_j$ appears positively in $C_i$. 
By definition of $\mathcal{I}$, $\mathcal{I} \models x_j$ and therefore $\mathcal{I} \models C_i$.
If $T_{ij} \nsubseteq E$ for all $j = 1,\ldots,m$, then $F_{ij} \subseteq E \subseteq E_{\Phi'}$, which implies that $x_j$ appears negatively in $C_i$, $x_j \notin \mathcal{I}$ and therefore $\mathcal{I} \models C_i$. Since $i$ was arbitrary we conclude $\mathcal{I} \models \Phi'$.

\end{proof}

\begin{lemma}[In NP]
\label{lemma:innp}
The short explanation decision problem is in NP.

\end{lemma}
\begin{proof}

Explanations are subsets of the input equations, therefore they are clearly polynomial in the problem size.
The congruence of two terms, i.e. verifying that a subset is actually an explanation, can be decided in $O(n \log(n))$ using for example the congruence closure algorithm presented in Section \ref{sec:algorithm}.

\end{proof}

Lemma \ref{lemma:nphardness} and \ref{lemma:innp} establish the main result of this section.

\begin{theorem}[NP - completeness]

The short explanation decision problem is NP- complete.

\end{theorem}