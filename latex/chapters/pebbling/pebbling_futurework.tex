\section{Future Work}
\label{sec:pebblingfuturework}

Most of the heuristics rely on basic information of the nodes.
More sophisticated combinations of characteristics of nodes and the proof as a whole as well as other sources of information could be used to come up with new and more successful heuristics.
For example information about the proof could already be used and stored during proof creation and conflict graph analysis.
Furthermore, different sequences of heuristics to decide ties could be evaluated.

The problem of constructing pebbling strategies with the pebbling number of the proof could be formulated as a constraint programming problem instance.
The encoding would be similar to the SAT encoding, but might not suffer from the high number of clauses.

Implementing the proposed methods directly into a SAT- or SMT solver would produce proofs with small space requirements right away without the need to post process it.
It would be interesting to see, whether our method can meet the high standards that solvers have to computation speed.

In the introduction the proof format DRAT was mentioned.
This proof format provides the possibility to output deletion information.
Comparing our deletion information with deletion information produced by some SAT or SMT solver, while keeping track of the maximum space required using this information, would show whether our method performs well w.r.t. methods used in solvers.