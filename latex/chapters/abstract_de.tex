\chapter*{Kurzfassung}

Diese Arbeit präsentiert zwei Methoden zur Komprimierung von formalen Beweisen.
Formale Beweise sind von gro{\ss}er Bedeutung in der modernen Informatik.
Sie k\"onnen verwendet werden um deduktive Systeme miteinander zu kombinieren. 
Ein Beispiel sind SAT- Solver \cite{Biere2009}, welche ob ihrer Effektivit\"at gerne f\"ur diverse Berechnungen verwendet werden.
Ein formaler Beweis kann als Zertifikat f\"ur die Korrektheit des Ergebnisses eines SAT- Solvers dienen.
Des Weiteren k\"onnen aus ihnen Informationen, wie etwa Interpolants \cite{McMill2005}, extrahiert werden, welche zur L\"osung eines Problems beitragen \cite{Hofferek2013}.

Formale Beweise sind typischerweise sehr gro{\ss}, siehe etwa \cite{Konev2014} f\"ur einen 13GB Beweis eines Falles der Erd\H{o}s Discrepancy Conjecture.
Bei solchen Beweisgr\"o{\ss}en sto{\ss}en Computersysteme an ihre Grenzen und deswegen ist es erforderlich Beweise zu komprimieren.
Unsere Arbeit pr\"asentiert zwei Methoden zur Beweiskomprimierung.

Die erste Methode entfernt Redundanzen im Gleichuntsteil von SMT-Beweisen.
Kongruenzbeweise schlie{\ss}en von einer Menge an Gleichungen auf neue Gleichungen mit der Vorraussetzung der vier Axiome: \emph{Reflexivit\"at}, \emph{Symmetrie}, \emph{Transitivit\"at} und \emph{Kongruenz}.
Beweise, die von SMT-Solvern erzeugt werden, schlie{\ss}en oft auf neue Gleichungen aus einer unn\"otig gro{\ss}en Menge.
Wir pr\"asentieren einen neuen Kongruenzschluss Algorithmus der Erkl\"arungen f\"ur gew\"unschte Gleichungen produziert und zeigen, dass diese Erkl\"arungen k\"urzer sind als solche in Benchmark- Beweisen.
Diese k\"urzeren Erkl\"arungen \"ubersetzen sich direkt in k\"urzere Teilbeweise und einen komprimierten Gesamtbeweis.
Des Weiteren beweisen wird, dass das Problem des Findens der k\"urzesten Erkl\"arung NP-Vollst\"andig ist.

Die zweite Methode untersucht die Speicherplatzanforderungen von Beweisen.
Beim Bearbeiten von Beweisen muss nicht der gesamte Beweis zu jeder Zeit im Speicher gelagert werden.
Teilbeweise werden erst in den Speicher geladen, wenn sie ben\"otigt werden und werden wieder aus diesem entfernt, sobald sie nicht mehr ben\"otigt werden.
In welcher Ordnung die Teilbeweise geladen werden, ist essentiell f\"ur die maximale Speicherplatzanforderung.
Wir wollen Ordnungen mit niedrigen Speicherplatzanforderungen mit Hilfe von zwei neuen Algorithmen und einer Reihe an Heuristiken konstruieren.