\section{Propositional Resolution Calculus}
\label{sec:Resolution}

A \emph{literal} is a propositional variable or the negation of a propositional variable. The
\emph{complement} of a literal $\ell$ is denoted $\dual{\ell}$ (i.e. for any propositional variable $p$,
$\dual{p} = \neg p$ and $\dual{\neg p} = p$). The set of all literals is denoted by $\mathcal{L}$. A
\emph{clause} is a set of literals. $\bot$ denotes the \emph{empty clause}.


\begin{definition}[Proof] 
\label{def:proof}
A \emph{proof} $\varphi$ is a tuple $\langle V,E,\n,\clause \rangle$, 
such that $\langle V,E \rangle$ is a labeled directed acyclic graph whose edges are labeled with literals $\ell \in \mathcal{L}$, i.e. $E \subseteq V \times \mathcal{L} \times V$, $\n \in V$, $\clause$ is a clause 
and one of the following holds:

\begin{enumerate}
	\item $V = \{\n\}, E = \emptyset$
	\item \label{enum:resCase} There are proofs $\varphi_L = \langle V_L, E_L, \n_L,\clause_L \rangle$ and $\varphi_R = \langle V_R, E_R, \n_R, \clause_R \rangle$ such that there exists a literal
				$\ell$, $\dual{\ell} \in \Gamma_L$ and $\ell \in \Gamma_R$, then
	    \begin{align*}
			\n & \text{ is a new node} \\
      V &= V_L \cup V_R \cup \{\n \} \\
      E &= E_L \cup E_R \cup
                    \left\{\n_L \xrightarrow{\dual{\ell}} \n , \n_R \xrightarrow{\ell} \n \right\} \\
     \Gamma &= \left( \clause_L \setminus \left\{ \dual{\ell} \right\} \right) \cup \left( \clause_R
                    \setminus \left\{ \ell \right\} \right)
    \end{align*}
		$\n$ is called the \emph{root} of $\varphi$ and $\clause$ its \emph{conclusion}.
		$\varphi_L$ and $\varphi_R$ are \emph{premises} of $\varphi$ and $\varphi$ is a \emph{child} of both $\varphi_L$ and $\varphi_R$.
		$\Gamma$ is called the \emph{resolvent} of $\Gamma_L$ and $\Gamma_R$ with \emph{pivot} $\ell$.
		A proof $\psi$ is a subproof of a proof $\varphi$, if there is a path from $\varphi$ to $\psi$ in the transitive closure of the premise relation.
		A subproof $\psi$ of $\varphi$ which has no premises is an \emph{axiom} of $\varphi$.
		$\Vertices{\varphi}$ and $\Axioms{\varphi}$ denote, respectively, the set of nodes and axioms of $\varphi$. $\Premises{\n}{\varphi}$ denotes the premises and $\Children{\n}{\varphi}$ the children of the subproof with root $\n$ in a proof $\varphi$. When a proof is represented graphically, the root is drawn at the bottom and the axioms at the top. The \emph{length} of a proof $\varphi$ is the number of nodes in $\Vertices{\varphi}$ and is denoted by $\plength{\varphi}$.
\qed
\end{enumerate}

\end{definition}

\noindent
Note that in case \ref{enum:resCase} of Definition \ref{def:proof} $V_L$ and $V_R$ are not required to be disjunct. 
Therefore the underlying structure of proofs are DAGs as opposed to simple trees. 
Modern SAT- and SMT-solvers, using techniques of conflict driven clause learning, produce proofs of such kind \cite{Bouton2009,Biere2009}.
The reuse of proof nodes plays a central role in proof compression \cite{Fontaine2011}.
Also note that a DAG corresponding to a proof by definition has exactly one sink, also called its root node.\\
For example a node of a proof $\langle V,E,\n,\clause \rangle$ will be meant to be some $s \in V$.
From hereon, if free from ambiguity, proofs and their underlying DAGs will not be distinguished. 
