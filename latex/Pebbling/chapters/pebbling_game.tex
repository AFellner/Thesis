\section{Pebbling Game}
\label{sec:pebbling-game}

%Todo: Discuss relation to resolution space complexity further
Pebbling games are played on graphs and pebbles are placed on nodes following the rules of the game.
The goal is to put a pebble on some target node.
Pebbling games were introduced in the 1970's to model programming language expressiveness \cite{Pippenger1980,Walker1973} and compiler construction \cite{Sethi1975}. 
More recently, pebbling games have been used to investigate various questions in parallel complexity \cite{Chan2013} and proof complexity \cite{Ben-Sasson2009,Esteban2001,Nordstroem2009}. 
They are used to obtain bounds for space and time requirements and trade-offs between the two measures \cite{EmdeBoas1979,Ben-Sasson2002}. 
From hereon \textit{to pebble} means to mark a node with a pebble and \textit{to unpebble} means to remove the mark of a node.

\begin{definition}[Bounded Pebbling Game]
\label{def:pebbling-game}
The \emph{Bounded Pebbling Game} is played by one player on a DAG $G = (V,E)$ with one distinguished node $s \in V$.
The goal of the game is to pebble $s$, respecting the following rules:
\begin{enumerate}
	\item \label{rule:premises} A node $\n$ is pebbleable \emph{iff} all predecessors of $\n$ in $G$ are pebbled and $\n$ is currently not pebbled.
	\item \label{rule:unpebbling} Pebbled nodes can be unpebbled at any time.
	\item \label{rule:onlyonce} Once a node has been unpebbled, it can not be pebbled again later.
\end{enumerate}
%Only pebbled nodes can be unpebbled and only unpebbled nodes can be pebbled.
The game is played in rounds.
Every round the player chooses a node $v \in V$, such that $v$ is pebbled or pebbleable.
The move of the player in this round is $p(v)$, if $v$ is pebbleable and $u(v)$ if $v$ is pebbled, where $p(.)$ and $v(.)$ correspond to pebbling and unpebbling a node respectively.
\qed
\end{definition}

Not that due to rule \ref{rule:premises} the move in each round is uniquely defined by the chosen node $v$.
The distinction of the two kinds of moves is just made for presentation purposes.
Also note that as a consequence of rule \ref{rule:premises}, pebbles can be put on nodes without predecessors at any time.
Playing the game on a proof $\varphi$ means to play the game on the underlying DAG with the distinguished node being the root of $\varphi$.

In this work we investigate space requirements when time requirements are fixed.
Fixing time is a design choice, see Section \ref{sec:pebblingchecking}, and it corresponds to rule \ref{rule:onlyonce}.
Including this rules sets a bound $O(|V|)$ for the number of rounds.

\begin{definition}[Strategy]
\label{def:strategy}
A \emph{pebbling strategy} $\sigma$ for the Bounded Pebbling Game, played on a DAG $G = (V,E)$ and distinguished node $s$, is a sequence of moves $(\sigma_1,\ldots,\sigma_n)$ of the player such that $\sigma_n = p(s)$.
\end{definition}

The following definition allows to measure how many pebbles are required to play the Bounded Pebbling Game on a given graph.

\begin{definition}[Pebbling number]
The \emph{pebbling number of a pebbling strategy} $(\sigma_1,\ldots,\sigma_n)$ is defined as 
$ max_{\indexIn{i}{1}{n}}|\{ \n \in V \mid \n \text{ is pebbled in round } i\}| $.
The \emph{pebbling number of a DAG $G$ and node $s$} is defined as the minimum pebbling number of all pebbling strategies for $G$ and $s$.
\end{definition}

Note that the definitions \ref{def:pebbling-game} and \ref{def:strategy} leave freedom when to do unpebbling moves.
With the aim of finding strategies with low pebbling numbers, there is a canonical way when to do these moves, as will be shown later.

\noindent
The Bounded Pebbling Game from definition \ref{def:pebbling-game} differs from the Black Pebbling Game discussed in \cite{Hertel2007,Pippenger1982} in two aspects. 
Firstly, the Black Pebbling Game does not include rule \ref{rule:onlyonce}. 
Excluding this rule allows for pebbling strategies with lower pebbling numbers (\cite{Sethi1975} has an example on page 1), which can have the cost of exponentially more moves \cite{EmdeBoas1979}.
Secondly, when pebbling a node in the Black Pebbling Game, one of its predecessors' pebbles can be used instead of a fresh pebble (i.e. a pebble can be moved). 
The trade-off when allowing to moving pebbles are discussed in \cite{EmdeBoas1979}. 
Deciding whether the pebbling number of a graph $G$ and node $s$ is smaller than $k$ is PSPACE-complete in the absence of rule \ref{rule:onlyonce} \cite{Gilbert1980} and NP-complete when rule \ref{rule:onlyonce} is included \cite{Sethi1975}.

