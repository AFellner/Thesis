\section{Introduction}

Proofs generated by SAT-solvers can be huge. 
Checking their correctness can not only take a long time but also consume a lot of memory. 
In an ongoing project for controller synthesis based on the extraction of interpolants from SMT-proofs \cite{Hofferek2013}, 
for example, post-processing a proof takes hours and may reach the limit of memory available today in a single node of a computer cluster (256GB). This issue is even more relevant in application scenarios in which the proof consumer, who is interested in independently checking the correctness of the proof, might have less available memory than the proof producer.
This is in part because, while the proof checker reads a usual proof file and checks the proof it contains, 
every proof node (containing a clause) that is loaded into memory has to be kept there until the end of the whole proof checking process, 
since the proof checker does not know whether a proof node will still need to be used and re-reading the proof file to reload and recheck proof nodes would be too time-consuming. 

To address this issue, recently proposed proof formats such as DRUP \cite{drup} and BDRUP \cite{bdrup} allow enriching a proof file with instructions that inform a proof checker when a proof node can be released from memory. Other proof formats, such as the TraceCheck format \cite{tracecheck} could also be enriched analogously. Such node deletion instructions can be added by a proof-generating SAT-solver during proof search in the periodic clean-up of its database of derived learned clauses; for every clause the SAT-solver deletes during this phase, this deletion can be recorded in the proof file. 

This paper explores the possibility of post-processing a proof in order to increase the amount of deletion instructions in the proof file. The more deletion instructions, the less memory the proof checker will need. Therefore, this \emph{deletion-during-proof-postprocessing} approach ought to be seen not as a replacement but rather as an independent complement to the \emph{deletion-during-proof-search} already performed by state-of-the-art proof-generating SAT-solvers.

The new methods proposed here exploit an analogy between proof checking and 
playing \emph{Pebbling Games} \cite{Kasai1979,Gilbert1980}. 
The particular version of pebbling game relevant for proof checking is defined precisely in Section \ref{sec:pebbling-game} and the analogy to proof checking is explained in detail in Section \ref{sec:pebblingchecking}. The proposed pebbling algorithms are greedy (Section \ref{sec:algorithms}) and based on heuristics (Section \ref{sec:heuristics}). As discussed in Sections \ref{sec:pebblingchecking} and \ref{sec:pebblingSAT}, approaches based on exhaustive enumeration or on encoding as a SAT problem would not fare well in practice.

The proof space compression algorithms described here are not restricted to proofs generated by SAT-solvers. They are general DAG pebbling algorithms, that could be applied to proofs represented in any calculus where proofs are directed acyclic graphs (including the special case of tree-like proofs). It is, nevertheless, in SAT and SMT that proofs tend to be largest and in most need of space compression. The underlying propositional resolution calculus (described in Section \ref{sec:Resolution}) satisfies the DAG requirement. The experiments (Section \ref{sec:experiments}) evaluate the proposed algorithms on thousands of SAT- and SMT-proofs.

