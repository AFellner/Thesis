\documentclass{llncs}

\usepackage{graphicx}
\usepackage{caption}
\usepackage{subcaption}
\captionsetup{compatibility=false}

\usepackage{xcolor}
\usepackage{enumitem,amsmath,amssymb}
\usepackage{breakurl}    % used for \url and \burl
\usepackage[linesnumbered,boxed,noline]{algorithm2e}
\def\defaultHypSeparation{\hskip.1in}

\usepackage{tikz}
%\usepackage{subfig}
\usepackage{array,booktabs,multirow}
\usepackage{placeins}

\usepackage{logictools}
\usepackage{prooftheory}
\usepackage{comment}
\usepackage{mathenvironments}
\usepackage{drawproof}

\DeclareMathOperator{\pebble}{P}
\DeclareMathOperator{\unpebble}{U}
\DeclareMathOperator{\pebbledAt}{pebbledAt}
\DeclareMathOperator{\peb}{peb}
\DeclareMathOperator{\free}{free}
\DeclareMathOperator{\ap}{ap}

\newcommand{\indexIn}[3]{\ensuremath{#1 \scriptstyle \in \{#2\ldots #3\} \displaystyle}}

\usetikzlibrary{patterns}

\renewcommand{\topfraction}{0.85}
\renewcommand{\textfraction}{0.1}
\renewcommand{\floatpagefraction}{0.75}

\newcommand{\defeq}{\mathrel{\mathop:}=}
\newcommand{\eqdef}{\mathrel{\mathop=}:}

\newcommand{\nodedistance}{0.55cm}
\newcommand{\nodedistanceThree}{0.6cm}
\newcommand{\nodedistanceTwo}{1.2cm}

\newcommand{\Vertices}[1]{V_{#1}}
\newcommand{\Edges}[1]{E_{#1}}
\newcommand{\Conclusion}[1]{\clause_{#1}}
\newcommand{\Premises}[2]{P_{#1}^{#2}}
\newcommand{\Children}[2]{C_{#1}^{#2}}
\newcommand{\Axioms}[1]{A_{#1}}

\newcommand{\axiom}[1]{\widehat{#1}}
\newcommand{\n}{v}
\newcommand{\raiz}[1]{\rho(#1)}

\newcommand{\pspace}[2]{s(#1,#2)}

\title{Greedy Pebbling: \\ 
Towards Proof Space Compression}

\author{
  Andreas Fellner 
  \thanks{Supported by the Google Summer of Code 2013 program.}
  \and 
  Bruno Woltzenlogel Paleo 
  \thanks{Supported by the Austrian Science Fund, project P24300.}
}

\authorrunning{A.\~Fellner \and B.\~Woltzenlogel Paleo}

\institute{
  \email{fellner.a@gmail.com} \ \ \ \email{bruno@logic.at} \\
  Theory and Logic Group \\
  Institute for Computer Languages \\
  Vienna University of Technology
}

\begin{document}

\maketitle

\setcounter{footnote}{0}

%%%%%%%%%%%%%%%%%%%%%%%%%%%%%%%%%%%%%%%%%
%%% Introduction %%%%%%%%%%%%%%%%%%%%%%%%
%%%%%%%%%%%%%%%%%%%%%%%%%%%%%%%%%%%%%%%%%
%\chapter{Introduction}
%\label{ch:introduction}
%%%%%%%%%%%%%%%%%%%%%%%%%%%%%%%%%%%%%%%%%

%\section{Introduction}

Proofs generated by SAT-solvers can be huge. 
Checking their correctness can not only take a long time but also consume a lot of memory. 
In an ongoing project for controller synthesis based on the extraction of interpolants from SMT-proofs \cite{Hofferek}, 
for example, post-processing a proof takes hours and may reach the limit of memory available today in a single node of a computer cluster (256GB). This issue is even more relevant in application scenarios in which the proof consumer, who is interested in independently checking the correctness of the proof, might have less available memory than the proof producer.
This is in part because, while the proof checker reads a usual proof file and checks the proof it contains, 
every proof node (containing a clause) that is loaded into memory has to be kept there until the end of the whole proof checking process, 
since the proof checker does not know whether a proof node will still need to be used and re-reading the proof file to reload and recheck proof nodes would be too time-consuming. 

To address this issue, recently proposed proof formats such as DRUP \cite{drup} and BDRUP \cite{bdrup} allow enriching a proof file with instructions that inform a proof checker when a proof node can be released from memory. Other proof formats, such as the TraceCheck format \cite{tracecheck} could also be enriched analogously. Such node deletion instructions can be added by a proof-generating SAT-solver during proof search in the periodic clean-up of its database of derived learned clauses; for every clause the SAT-solver deletes during this phase, this deletion can be recorded in the proof file. 

This paper explores the possibility of post-processing a proof in order to increase the amount of deletion instructions in the proof file. The more deletion instructions, the less memory the proof checker will need. Therefore, this \emph{deletion-during-proof-postprocessing} approach ought to be seen not as a replacement but rather as an independent complement to the \emph{deletion-during-proof-search} already performed by state-of-the-art proof-generating SAT-solvers.

The new methods proposed here exploit an analogy between proof checking and 
playing \emph{Pebbling Games} \cite{kasai1979classes,gilbert1980pebbling}. 
The particular version of pebbling game relevant for proof checking is defined precisely in Section \ref{sec:pebbling-game} and the analogy to proof checking is explained in detail in Section \ref{sec:pebblingchecking}. The proposed pebbling algorithms are greedy (Section \ref{sec:algorithms}) and based on heuristics (Section \ref{sec:heuristics}). As discussed in Sections \ref{sec:pebblingchecking} and \ref{sec:PebblingAsSat}, approaches based on exhaustive enumeration or on encoding as a \emph{Sat} problem would not fare well in practice.

The proof space compression algorithms described here are not restricted to proofs generated by SAT-solvers. They are general DAG pebbling algorithms, that could be applied to proofs represented in any calculus where proofs are directed acyclic graphs (including the special case of tree-like proofs). It is, nevertheless, in \emph{Sat} and \emph{SMT} that proofs tend to be largest and in most need of space compression. The underlying propositional resolution calculus (described in Section \ref{sec:Resolution}) satisfies the DAG requirement. The experiments (Section \ref{sec:exp}) evaluate the proposed algorithms on thousands of SAT- and SMT-proofs.



%%%%%%%%%%%%%%%%%%%%%%%%%%%%%%%%%%%%%%%%%
%%% Resolution %%%%%%%%%%%%%%%%%%%%%%%%%%
%%%%%%%%%%%%%%%%%%%%%%%%%%%%%%%%%%%%%%%%%
%\chapter{Resolution}
%\label{ch:resolution}
%%%%%%%%%%%%%%%%%%%%%%%%%%%%%%%%%%%%%%%%%

%%\chapter*{Resolution}

\section*{Congruence Resolution}

The proofs considered in this work are resolution proofs \cite{TODO: resolution} extended by the axioms of congruence.
In this chapter we define the calculus 

Let $\mathcal{C}$ be a finite set of constant symbols and let $f$ be a binary function symbol.
The set of terms $\mathcal{T}$ is defined recursively.
\begin{align*}
	\mathcal{T}_0 &:= \mathcal{C} \\
	\mathcal{T}_i &:= \mathcal{T}_{i-1} \cup \{f(t_1,t_2) \mid t_1, t_2 \in \mathcal{T}_{i-1}\} \\
	\mathcal{T} &:= \bigcup_{n \in \mathbb{N}} \mathcal{T}_n
\end{align*}
Let $\mathcal{Q}_{\mathcal{T}} = \{t_1 = t_2 \mid t_1, t_2 \in \mathcal{T}$ be the set of equations for a set of terms $\mathcal{T}$
Let $V$ be a finite set of propositional variables.
The set of equality atoms $\mathcal{E}$ is defined as $V \cup Q_{\mathcal{T}}$
An equality literal $\ell_\mathcal{T}$ is an equality atom $e$ or a negated equality atom $\neg e$.
We will abbreviate write $\neg (t_1 = t_2)$ by $t_1 \neq t_2$.
An equality clause is a set of equality literals.
As usual a clause is interpreted as the disjunction of its literals and a set of clauses is interpreted as the conjunction its clauses.

The axioms of congruence $EqAxioms$ for some set of terms $\mathcal{T}$ is defined as $R \cup S \cup T \cup C$ where

\begin{align*}
	R &= \{ t = t \mid t \in \mathcal{T}\} \\
	S &= \{ t_1 \neq t_2, t_2 = t_1 \mid t_1, t_2 \in \mathcal{T} \} \\
	T &= \{ t_1 \neq t_2, t_2 \neq t_3, t_1 = t_3 \mid t_1, t_2, t_3 \in \mathcal{T} \} \\
	C &= \{ t_1 \neq t_3, t_2 \neq t_4, f(t_1,t_2) = f(t_3,t_4) \mid t_1, t_2, t_3, t_4 \in \mathcal{T} \} 
\end{align*}

Note that every congruence axiom has one positive equality literal.
From now on we will omit the set of terms $\mathcal{T}$ if it is clear from context.

Next we will define the resolution calculus extended by congruence axioms.
Let $\ell$ be an equality literal and $C_1$, $C_2$ be equality clauses such that $\ell \in C_1$ and $\neg \ell \in C_2$.
The clause $C_1 \setminus \{\ell\} \cup C_2 \setminus \{\neg \ell\}$ is the resolvent of $C_1$ and $C_2$ with pivot $\ell$.

Let $F = \{C_1, \ldots, C_n\}$ be a set of clauses.
The notion of a congruence derivation for $F$ is defined inductively.
The sequence $\langle C_1, \ldots, C_n\rangle$ is a congruence derivation for $F$.
If $\langle C_1, \ldots, C_m\rangle$ is a congruence derivation for $F$ then $\langle C_1, \ldots, C_{m+1} \rangle$ is a congruence derivation for $F$ if $C_{m+1} \in EqAxioms$ or $C_{m+1}$ is a resolvent of $C_i$ and $C_j$ with $1 \leq i,j \leq m$.
A congruence derivation containing the empty clause is a congruence refutation.

Let $D = \langle C_1, \ldots, C_m\rangle$ be a congruence derivation.
The longest subsequence $\langle C_{i_1}, \ldots, C_{i_k}\rangle$ of $D$, such that $\{C_{i_1}, \ldots, C_{i_k}\} \subseteq EqAxioms$ is called the equality reasoning part of $D$.
%The corresponding sequence of positive equality literals $\langle e_{i_1}, \ldots, e_{i_k}\rangle$ 


%%%%%%%%%%%%%%%%%%%%%%%%%%%%%%%%%%%%%%%%%
%%% Pebbling Game %%%%%%%%%%%%%%%%%%%%%%%
%%%%%%%%%%%%%%%%%%%%%%%%%%%%%%%%%%%%%%%%%
%\chapter{Pebbling Game}
%\label{ch:pebbling-game}
%%%%%%%%%%%%%%%%%%%%%%%%%%%%%%%%%%%%%%%%%

%\section{Pebbling Game}
\label{sec:pebbling-game}

%Todo: Discuss relation to resolution space complexity further
Pebbling games are played on graphs and pebbles are placed on nodes following the rules of the game.
The goal is to put a pebble on some target node.
Pebbling games were introduced in the 1970's to model programming language expressiveness \cite{Pippenger1980,Walker1973} and compiler construction \cite{Sethi1975}. 
More recently, pebbling games have been used to investigate various questions in parallel complexity \cite{Chan2013} and proof complexity \cite{Ben-Sasson2009,Esteban2001,Nordstroem2009}. 
They are used to obtain bounds for space and time requirements and trade-offs between the two measures \cite{EmdeBoas1979,Ben-Sasson2002}. 
From hereon \textit{to pebble} means to mark a node with a pebble and \textit{to unpebble} means to remove the mark of a node.

\begin{definition}[Bounded Pebbling Game]
\label{def:pebbling-game}
The \emph{Bounded Pebbling Game} is played by one player on a DAG $G = (V,E)$ with one distinguished node $s \in V$.
The goal of the game is to pebble $s$, respecting the following rules:
\begin{enumerate}
	\item \label{rule:premises} A node $\n$ is pebbleable \emph{iff} all predecessors of $\n$ in $G$ are pebbled and $\n$ is currently not pebbled.
	\item \label{rule:unpebbling} Pebbled nodes can be unpebbled at any time.
	\item \label{rule:onlyonce} Once a node has been unpebbled, it can not be pebbled again later.
\end{enumerate}
%Only pebbled nodes can be unpebbled and only unpebbled nodes can be pebbled.
The game is played in rounds.
Every round the player chooses a node $v \in V$, such that $v$ is pebbled or pebbleable.
The move of the player in this round is $p(v)$, if $v$ is pebbleable and $u(v)$ if $v$ is pebbled, where $p(.)$ and $v(.)$ correspond to pebbling and unpebbling a node respectively.
\qed
\end{definition}

Not that due to rule \ref{rule:premises} the move in each round is uniquely defined by the chosen node $v$.
The distinction of the two kinds of moves is just made for presentation purposes.
Also note that as a consequence of rule \ref{rule:premises}, pebbles can be put on nodes without predecessors at any time.
Playing the game on a proof $\varphi$ means to play the game on the underlying DAG with the distinguished node being the root of $\varphi$.

In this work we investigate space requirements when time requirements are fixed.
Fixing time is a design choice, see Section \ref{sec:pebblingchecking}, and it corresponds to rule \ref{rule:onlyonce}.
Including this rules sets a bound $O(|V|)$ for the number of rounds.

\begin{definition}[Strategy]
\label{def:strategy}
A \emph{pebbling strategy} $\sigma$ for the Bounded Pebbling Game, played on a DAG $G = (V,E)$ and distinguished node $s$, is a sequence of moves $(\sigma_1,\ldots,\sigma_n)$ of the player such that $\sigma_n = p(s)$.
\end{definition}

The following definition allows to measure how many pebbles are required to play the Bounded Pebbling Game on a given graph.

\begin{definition}[Pebbling number]
The \emph{pebbling number of a pebbling strategy} $(\sigma_1,\ldots,\sigma_n)$ is defined as 
$ max_{\indexIn{i}{1}{n}}|\{ \n \in V \mid \n \text{ is pebbled in round } i\}| $.
The \emph{pebbling number of a DAG $G$ and node $s$} is defined as the minimum pebbling number of all pebbling strategies for $G$ and $s$.
\end{definition}

Note that the definitions \ref{def:pebbling-game} and \ref{def:strategy} leave freedom when to do unpebbling moves.
With the aim of finding strategies with low pebbling numbers, there is a canonical way when to do these moves, as will be shown later.

\noindent
The Bounded Pebbling Game from definition \ref{def:pebbling-game} differs from the Black Pebbling Game discussed in \cite{Hertel2007,Pippenger1982} in two aspects. 
Firstly, the Black Pebbling Game does not include rule \ref{rule:onlyonce}. 
Excluding this rule allows for pebbling strategies with lower pebbling numbers (\cite{Sethi1975} has an example on page 1), which can have the cost of exponentially more moves \cite{EmdeBoas1979}.
Secondly, when pebbling a node in the Black Pebbling Game, one of its predecessors' pebbles can be used instead of a fresh pebble (i.e. a pebble can be moved). 
The trade-off when allowing to moving pebbles are discussed in \cite{EmdeBoas1979}. 
Deciding whether the pebbling number of a graph $G$ and node $s$ is smaller than $k$ is PSPACE-complete in the absence of rule \ref{rule:onlyonce} \cite{Gilbert1980} and NP-complete when rule \ref{rule:onlyonce} is included \cite{Sethi1975}.



%%%%%%%%%%%%%%%%%%%%%%%%%%%%%%%%%%%%%%%%%
%%% Pebbling and Proof Checking %%%%%%%%%
%%%%%%%%%%%%%%%%%%%%%%%%%%%%%%%%%%%%%%%%%
%\chapter{Pebbling and Proof Processing}
%\label{ch:pebblingprocessing}
%%%%%%%%%%%%%%%%%%%%%%%%%%%%%%%%%%%%%%%%%

%\section{Pebbling and Proof Processing}
\label{sec:pebblingchecking}

The problem of processing a proof with minimal memory consumption is analogous to the problem of finding a pebbling strategy with minimal pebbling number.
Proof processing could be checking its correctness, manipulating it or extracting information from it.
The following definition makes the notion of proof processing formal.

\begin{definition}[Proof Processing]
\label{def:proof-processing}

Let $\varphi$ be a proof with nodes $V$ and $T$ be an arbitrary set.
Let $S = T \cup \{\bot\}$ where $\bot$ is a new symbol.
We call a partial function $f: U \times S \times S \rightarrow T$ a \emph{processing function}.
The partial \emph{apply function} $\ap: V \times T^{U\times S \times S} \rightarrow T$ is defined recursively as follows.
$$
\ap(\n,f) = \Bigg\{
\begin{array}{ll}
	f(\n,\ap(pr_1),\ap(pr_2)) &\text{ if } \n \text{ has premises } pr_1 \text{ and } pr_2\\
	f(\n,\bot,\bot) &\text{ if } $\n$ \text{ is an axiom}\\
	\text{ undefined }&\text{ otherwise}
\end{array}
$$

\emph{Processing a node} $\n$ with some processing function $f$ means computing the value $\ap(\n,f)$.
\emph{Processing a proof} means to process its root node.
\qed
\end{definition}

\begin{example}

Checking the correctness of a proof (i.e. checking for the absence of faulty resolution steps) can be checked in terms of the following processing function with $T = \{true,false\}$.
$$
f(\n,w_1,w_2) = \left\{
\begin{array}{ll}
	true & \text{ if $w_1 = w_2 = \bot$} \\
	w_1 \wedge w_2 &\text{ if the conclusion of $\n$ is a resolvent}\\
								 &\quad \text{ of the conclusions of its premises} \\
	false & \text{ otherwise}
\end{array}
\right.
$$
Processing a proof with this processing function yields $true$ \emph{iff} the proof is a correct resolution proof.
\qed
\end{example}

In Section \ref{sec:pebbling-game} it was pointed out that strategies with minimal pebbling numbers may require to play exponentially many rounds.
Every round of the game corresponds to an I/O operation and, if the action of the player is to pebble a node, the processing of the node.
The goal of proof compression is to reduce the hardness of proof processing, therefore requiring exponentially many I/O operations and processing steps is not a viable option.
That is the reason why we chose the static pebbling game for our purpose.
In the static pebbling game the number of rounds is essentially the number of nodes.

In order to process a node, the results of processing its premises are used and therefore have to be stored in memory.
The requirement of having premises in memory corresponds to rule \ref{rule:premises} of the static pebbling game. 
Processing a node and I/O operations are typically more expensive than extra memory consumption, therefore in our setting every node can be processed only once, which corresponds to rule \ref{rule:onlyonce}.
A node that has been processed can be removed from memory, which corresponds to rule \ref{rule:unpebbling}.
Note that removing a node and its results too early in combination with \ref{rule:onlyonce} makes it impossible to process the whole proof.
The best moment for removing a node from memory is canonical, as shown later in this chapter.

Definition \ref{def:proof-processing} leaves freedom in what order to process nodes.
The order in which nodes are processed is essential for the memory consumption, just like the order of pebbling nodes in the pebbling game is essential for the pebbling number.
The following definition allows us to relate pebbling strategies with orderings of nodes.

\begin{definition}[Topological Order]
\label{def:topological-order}
A topological order of a proof $\varphi$ is a total order relation $\prec$ on $\Vertices{\varphi}$, such that 
$\text{for all } \n \in \Vertices{\varphi} \text{, for all } p \in \Premises{\n}{\varphi}:
p \prec v$.
A pebbling strategy $\sigma = (\sigma_1,\ldots,\sigma_n)$ \emph{respects} a topological order $\prec$ if $j < i$ \emph{iff} $\sigma_j \prec \sigma_i$.
\qed
\end{definition}

A topological order $\prec$ of a proof $\varphi$ can be represented as a sequence $(v_1,\dots,v_n)$ of proof nodes, by defining $\prec \defeq \{(v_i,v_j) \mid 1 \leq i < j \leq n\}$. 
This sequence can be interpreted as a particular pebbling strategy for the game played on $\varphi$.
The requirement of topological orders to order premises higher than their children corresponds to rule \ref{rule:premises} of the static pebbling game.
The antisymmetry together with the fact that $V = \{v_1,\dots,v_n\}$ corresponds to rule \ref{rule:onlyonce}.
The only thing missing are the unpebbling moves.
The following theorem states that unpebbling moves are given implicit, when the goal is to find strategies with small pebbling numbers.

%TODO: remove the ugly k
\begin{definition}[Canonical Topological Pebbling Strategy]
The \emph{canonical topological pebbling strategy} $\sigma$ for a proof $\varphi$, its root node $s$ and a topological order $\prec$ represented as a sequence $(v_1,\dots,v_n)$ is recursively:
$$
\begin{array}{ll}
\sigma_1 = v_1 & \\
\sigma_i = v_j\text{ such that } & \text{for all }v \in \Premises{v_j}{\varphi}:\text{ there exists }k < i, \sigma_k = v \\
                                 & \text{or } j = min(k \mid \text{ for all }l < i: \sigma_l \neq v_k
\end{array}
$$
\qed
\end{definition}

The following theorem shows that unpebbling moves can be omitted from strategies for the static pebbling game, when the goal is to produce strategies with low pebbling numbers.

\begin{theorem}
\label{theorem:canonical}
The canonical pebbling strategy has the minimum pebbling number among all pebbling strategies that respect the topological order $\prec$.
\end{theorem}
\begin{proof} (Sketch)
All the pebbling strategies respecting $\prec$ differ only w.r.t. their unpebbling moves.
Consider the unpebbling of an arbitrary node $v$ in the canonical pebbling strategy. Unpebbling it later could only possibly increase the pebble number. To reduce the pebble number, $v$ would have to be unpebbled earlier than some preceding pebbling move. But, by definition of canonical pebbling strategy, the immediately preceding pebbling move pebbles the last child of $v$ w.r.t. $\prec$. Therefore, unpebbling $v$ earlier would make it impossible for its last child to be pebbled later without violating the rules of the game.
\qed
\end{proof}

As a consequence of Theorem \ref{theorem:canonical} is that finding pebbling strategies with low pebbling numbers can be reduced to constructing topological orders.
The memory required to process a proof using some topological order can be measured by the pebbling number of the pebbling strategy corresponding to the order.

\begin{definition}[Space]
\label{def:space measure}
The \emph{space} $\pspace{\varphi}{\prec}$ 
of a proof $\varphi$ and a topological order $\prec$ is the pebbling number of the canonical topological pebbling strategy of $\varphi$, its root and $\prec$.
\qed
\end{definition}

The problem of compressing the space of a proof $\varphi$ and a topological order $\prec$ is the problem of finding another topological order $\prec'$ such that $\pspace{\varphi}{\prec'} < \pspace{\varphi}{\prec}$. The following theorem shows that the number of possible topological orders is very large; hence, enumeration is not a feasible option when trying to find a good topological order.

\begin{theorem}
\label{theorem:enumeration}
There is a sequence of proofs $(\varphi_1,\ldots,\varphi_m,\ldots)$ such that $\plength{\varphi_m} \in O(m)$ and $|T(\varphi_m)| \in \Omega(m!)$, where $T(\varphi_m)$ is the set of possible topological orders for $\varphi_m$.
\end{theorem}
\begin{proof}
Let $\varphi_m$ be a perfect binary tree with $m$ axioms. Clearly, $\plength{\varphi_m} = 2m-1$.
Let $(\n_1,\ldots,\n_n)$ be a topological order for $\varphi_m$. 
Let $\Axioms{\varphi} = \{\n_{k_1},\ldots,\n_{k_m}\}$, then $(\n_{k_1},\ldots,\n_{k_m},\n_{l_1},\ldots,\n_{l_{n-m}})$, where $(l_1,\ldots,l_{n-m}) = (1,\ldots,n) \setminus (k_1,\ldots,k_m)$, is a topological order as well. 
Likewise, $(\n_{\pi({k_1})},\ldots,\n_{\pi({k_m})},\n_{l_1},\ldots,\n_{l_{n-m}})$ is a topological order, for every permutation $\pi$ of $\{k_1,\ldots,k_m\}$. There are $m!$ such permutations, so the overall number of topological orders is at least factorial in $m$ (and also in $n$).
\end{proof}



%%%%%%%%%%%%%%%%%%%%%%%%%%%%%%%%%%%%%%%%%
%%% Pebbling as a Satisfiability Problem 
%%%%%%%%%%%%%%%%%%%%%%%%%%%%%%%%%%%%%%%%%
%\chapter{Pebbling as a Satisfiability Problem}
%\label{ch:pebblingSAT}
%%%%%%%%%%%%%%%%%%%%%%%%%%%%%%%%%%%%%%%%%

%\section{Pebbling as a Satisfiability Problem}
\label{sec:pebblingSAT}

To find the pebble number of a proof, the question whether the proof can be pebbled using no more than $k$ pebbles can be encoded as a propositional satisfiability problem.
In this section let $\varphi$ be a proof with nodes $\n_1,\ldots,\n_n$ and let $\n_n$ be its root. 
Due to rule \ref{rule:onlyonce} of the Bounded Pebbling Game, the number of moves that pebble nodes is exactly $n$ and due to theorem \ref{theorem:canonical} determining the order of these moves is enough to define a strategy. 

In our SAT encoding, for every $x \in \{1,\ldots,k\}$, every $j \in \{1,\ldots,n\}$ and every $t \in \{0,\ldots,n\}$ there is a propositional variable $p_{x,j,t}$. 
The variable $p_{x,j,t}$ being mapped to $\top$ by a valuation is interpreted as the fact that in the $t$'th round of the game node $v_j$ is marked with pebble $x$.
Round $0$ is interpreted as the initial setting of the game before any move has been done.

For pebbling strategies, it is not relevant which of the $k$ pebbles is on a node.
Therefore one could also think of an encoding where true variables simply mean that a node is pebbled.
However, such an encoding would require exponentially many clauses (in $k$) when limiting the number of pebbles used in a round.

\begin{definition}[Pebbling SAT encoding]
%The following constraints, combined conjunctively, are satisfiable \textit{iff} there is a pebbling strategy $\sigma$ for $\varphi$ with pebbling number smaller or equal $k$. 
%In case the formula is satisfiable, a pebbling strategy can be read off from any satisfying assignment.
The conjunction of the following four constraints expresses the existence of a pebbling strategy for $\varphi$ with pebbling number smaller or equal $k$.

\begin{enumerate}
	\item The root is pebbled in the last round
				$$\Psi_1 = \bigvee_{x = 1}^k p_{x,n,n}$$
				
	\item No node is pebbled initially\\
				$$\Psi_2 = \bigwedge_{x = 1}^k \bigwedge_{j = 1}^n \left(\neg p_{x,j,0} \right)$$

	\item A pebble can only be on one node in one round
				$$\Psi_3 = \bigwedge_{x = 1}^k \bigwedge_{j = 1}^n \bigwedge_{t = 1}^n \left( p_{x,j,t} \rightarrow \bigwedge_{i = 1, i \neq j}^n \neg p_{x,i,t} \right)$$ 
				
	\item \label{c:pebble} For pebbling a node, its premises have to be pebbled the round before and only one node is being pebbled each round.\\
				\begin{align*}
					\Psi_4 = \bigwedge_{x = 1}^k \bigwedge_{j = 1}^n \bigwedge_{t = 1}^n & \Bigg( \left(\neg p_{x,j,t} \wedge p_{x,j,(t+1)} \right)\rightarrow \\
					&\bigg( \bigwedge_{i \in \Premises{j}{\varphi}} \bigvee_{y = 1, y \neq x}^k p_{y,i,t} \bigg) \wedge 
					\bigg( \bigwedge_{i = 1}^n \bigwedge_{y = 1, y \neq x}^k \neg \left( \neg p_{y,i,t} \wedge p_{y,i,(t+1)} \right) \bigg) \Bigg)
				\end{align*}
				
\end{enumerate}

The sets $\Axioms{\varphi}$ and $\Premises{j}{\varphi}$ are to be understood as sets of indices of the respective nodes.

\end{definition}

\noindent
This encoding is polynomial, both in $n$ and $k$. However constraint \ref{c:pebble} accounts to $O(n^3*k^2)$ clauses. 
Even small resolution proofs have more than $1000$ nodes and pebble numbers bigger than $100$, which adds up to $10^{13}$ clauses for constraint \ref{c:pebble} alone. 
Therefore, although theoretically possible to play the pebbling game via SAT-solving, this is practically infeasible for compressing proof space.
The following theorem proves the correctness of the encoding.

\begin{theorem}[Correctness of pebbling SAT encoding]

$\Psi = \Psi_1 \wedge \Psi_2 \wedge \Psi_3 \wedge \Psi_4$ is satisfiable if and only if there exists a pebbling strategy using no more than $k$ pebbles

\end{theorem}

\begin{proof}

\emph{Suppose $\Psi$ is satisfiable} and let $\mathcal{I}$ be a satisfying variable assignment in form of the set of true variables.
We will use $P(x,j,t)$ as an abbreviation for $p_{x,j,(t-1)} \notin \mathcal{I}$ and $p_{x,j,t} \in \mathcal{I}$.
Since $\mathcal{I}$ satisfies $\Psi_3$, in $P(x,j,t)$ $x$ is uniquely defined by $j$ and $t$ and we can write $P(j,t)$ instead.
We will prove the following assertion.
For every $t \in \{1,\ldots,n\}$ there exists exactly one $j \in \{1,\ldots,n\}$ such that $P(j,t)$.

%Every node, expect the root itself, is a recursive ancestor of the proof root.
$\Psi_1$ states that the root $v_n$ has to be pebbled in the last round and $\Psi_2$ states that no node is pebbled initially.
So for $n$ there has to be a $t \in \{1,\ldots,n\}$ such that $P(n,t)$.
$\mathcal{I}$ satisfies $\Psi_4$, therefore for every predecessor of $v_j$ of $v_n$ there exists $x \in \{1,\ldots,k\}$ such that $p_{x,j,(t-1)}$.
Using the same argument for $v_j$ as for $v_n$ there has to be a $t' \in \{1,\ldots,(t-1)\}$ such that $P(j,t')$.
Every node of the proof is a recursive ancestor of the root, therefore for every $j \in \{1,\ldots,n\}$ there exists at least one $t \in \{1,\ldots,n\}$ such that $P(n,t)$.
For every $t \in \{1,\ldots,n\}$, $\Psi_4$ ensures that if $P(n,t)$ then there is no $i \in \{1,\ldots,n\}, i \neq j$ such that $P(i,t)$, which proves the assertion.
The assertion implies the existence of a bijection $\tau : \{1,\ldots,n\} \rightarrow \{v_1,\ldots,v_n\}$ such that $\tau(n) = v_n$ and $\tau(t) = j$ \emph{iff} $P(j,t)$.
Therefore $\sigma := \{\tau(1),\ldots,\tau(n)\}$ is well defined.
$\sigma$ is a pebbling strategy, because $\tau(n) = v_n$, rule \ref{rule:premises} is obeyed because of $\Psi_4$, rule \ref{rule:unpebbling} is obeyed, because unpebbling moves are given implicitly (see Theorem \ref{theorem:canonical}) and rule \ref{rule:onlyonce} is obeyed because $\tau$ is a bijection.
$\Psi_3$ being satisfied ensures that $\sigma$ uses no more than $k$ pebbles.

\emph{Suppose there is a pebbling strategy $\sigma$ using no more than $k$ pebbles}. Let the function $\free: \{1,\ldots,n\} \rightarrow 2^{\{1,\ldots,k\}} \setminus \emptyset$ be defined recursively as follows and $\peb(t) = \min(\free(t)).$

$$
\free(t) = \left\{
  \begin{array}{ll}
    \{1,\ldots,k\} & : t = 1\\
    \free(t-1) \setminus \{\peb(t-1)\} \quad \cup & : otherwise\\
		\Big\{\peb(s) \mid \sigma_{s} \in \Premises{\sigma_{t-1}}{\varphi}, s \in \{1,\ldots,t-2\} \text{ and for all } v \in \Children{\sigma_{s}}{\varphi}&\\
		\text{ there exists } r \in \{1,\ldots,t-1\}: \sigma_{r} = v \Big\}&
  \end{array}
\right.
$$

Intuitively, $\free(.)$ keeps track of the unused pebbles in each round.
If a pebble is placed on a node, it is not free anymore.
Pebbles are made free again by unpebbling moves, which correspond to the second set in the recursive definition of $\free(.)$.
Since $\sigma$ uses no more than $k$ pebbles, $\free(.)$ is well defined.

Let $\mathcal{I}$ be a set of variables of $\Psi$ defined as follows.
$p_{x,j,t} \in \mathcal{I}$ \emph{iff} $t > 0$ and there exists $s \in \{1,\ldots,t\}$ such that $\peb(s) = x$, $\sigma_s = v_j$ and for all $r \in \{s+1,\ldots,t\}: x \notin \free(r)$.
$\mathcal{I}$ is a satisfying assignment for $\Psi$.
$\Psi_1$ is satisfied, because $\sigma_n = v_n$, therefore trivially $p_{\peb(n),n,n} \in \mathcal{I}$.
Clearly $\Psi_2$ is satisfied by $\mathcal{I}$ as no variables with $t = 0$ are included in $\mathcal{I}$.
To see that $\Psi_3$ is satisfied, suppose there exist $x,t,i,j$ such that $i \neq j$ and $\{p_{x,j,t},p_{x,i,t}\} \subseteq \mathcal{I}$.
Then by definition of $\mathcal{I}$ there exist unique $t_1$ and $t_2$ such that $\peb(t_1) = x, \sigma_{t_1} = v_j$ and $\peb(t_2) = x, \sigma_{t_2} = v_i$.
From $i \neq j$ follows $v_i \neq v_j$, therefore $t_1 \neq t_2$ w.l.o.g. suppose $t_1 > t_2$.
From $\peb(t_2) = x$, $p_{x,i,t} \in \mathcal{I}$ and $t \geq t_1 > t_2$ follows $x \notin \free(t_1)$, which is a contradiction to $\peb(t_1) = x$.
Let $P(x,j,t)$ be defined as above. Then from $P(x,j,t)$ follows $\peb(t) = x$ and $\sigma_t = v_j$.
Rule \ref{rule:premises} of the Bounded Pebbling Game ensures that there if $P(x,j,t)$ is true, then there exists a $y \{1,\ldots,k\} \setminus \{x\}$ such that $p_{y,i,t-1} \in \mathcal{I}$.
Suppose $P(x,j,t)$ and $P(y,i,t)$ both hold for some $t$, $x \neq y$ and $i \neq j$, then $y = \peb(t) = x$ and $v_j = \sigma_t = v_i$ are both contradictions. 
Therefore also $\Psi_4$ is satisfied by $\mathcal{I}$.

\end{proof}

%%%%%%%%%%%%%%%%%%%%%%%%%%%%%%%%%%%%%%%%%
%%% Greedy Pebbling Algorithms %%%%%%%%%%
%%%%%%%%%%%%%%%%%%%%%%%%%%%%%%%%%%%%%%%%%
%\chapter{Greedy Pebbling Algorithms}
%\label{ch:algorithms}
%%%%%%%%%%%%%%%%%%%%%%%%%%%%%%%%%%%%%%%%%

%\section{Greedy Pebbling Algorithms}
\label{sec:algorithms}

Theorem \ref{theorem:enumeration} and the remarks in the end of section \ref{sec:PebblingAsSat} indicate that obtaining an optimal topological order either by enumerating topological orders or by encoding the problem as a satisfiability problem is impractical. 
This section presents two greedy algorithms that aim at finding good though not necessarily optimal topological order. 
They are both parameterized by some heuristic described in Section \ref{sec:heuristics}, but differ in the traversal direction in which the algorithms operate on proofs.

\subsection{Top-Down Pebbling}

\algo{Top-Down} Pebbling (Algorithm \ref{algo:TDpebbling}) constructs a topological order of a proof $\varphi$ by traversing it from its axioms to its root node.
This approach closely corresponds to how a human would play the pebbling game. 
A human would look at the nodes that are available for pebbling in the current round of the game, choose one of them to pebble and remove pebbles if possible.
Similarly the algorithm keeps track of pebblable nodes in a set $N$, initialized as $\Axioms{\varphi}$.
When a node $\n$ is pebbled, it is removed from $N$ and added to the sequence representing the topological order. The children of $\n$ that become pebbleable are added to $N$.
When $N$ becomes empty, all nodes have been pebbled once and a topological order has been found.


\begin{algorithm}[h]
  \KwIn{proof $\varphi$}
  \KwOut{sequence of nodes $S$ representing a topological order $\prec$ of $\varphi$}
	
	$S = ()$; \tcp*[f]{the empty sequence} \\
	$N = \Axioms{\varphi}$; \tcp*[f]{initialize pebbleable nodes with Axioms}
	
  \While{$N$ is not empty}{
    choose $\n \in N$ heuristically\;
		$S = S ::: (\n)$; \tcp*[f]{$:::$ is the concatenation of sequences}\\
		$N = N \setminus \{\n\}$\;
		\For(\tcp*[f]{check whether $c$ is now pebbleable}){\KwSty{each} $c \in \Children{\n}{\varphi}$}{ 
			\If{$\forall p \in \Premises{c}{\varphi}: p \in S$}
					{$N = N \cup \{c\}$\;}
					}
  }
	
	\Return $S$\;
	
  \caption[.]{\FuncSty{Top-Down Pebbling}}
  \label{algo:TDpebbling}
\end{algorithm}

\begin{example}
\label{example:topdown}

\begin{figure}[tb]
	\makebox[\textwidth][c]{
	\begin{minipage}{.33\textwidth}
		\begin{tikzpicture}[node distance=\nodedistance]
			\proofnodeBW{root};
			\proofnodeBW[above left of = root,xshift=-\nodedistance]{n7};
			\proofnodeBW[above right of = n7,xshift=\nodedistance]{n6};
			\proofnodeBW[above left of = n7]{n3};
			\proofnodeBW[above right of = root,xshift=\nodedistance]{n12};
			\proofnodeBW[above right of = n12]{n10};
			\withchildrenBW{n3}{n1}{n2};
			\withchildrenBW{n10}{n8}{n9};
			\proofnodeBW[above right of = n2]{n4};
			\proofnodeBW[above left of = n8]{n5};
			\drawchildren{n6}{n4}{n5};
			\withchildrenBW{n4}{e1}{e2};
			\withchildrenBW{n5}{e3}{e4};
			\drawchildren{root}{n7}{n12};
			\drawchildren{n7}{n3}{n6};
			\drawchildren{n12}{n6}{n10};
			\blacknode{n3};
			\node [yshift = 1.5mm] (cap1) at (n1.north) {\small{1}};
			\node [yshift = 1.5mm] (cap2) at (n2.north) {\small{2}};
			\node [yshift = 1.5mm] (cap3) at (n3.north) {\small{3}};
			\node [yshift = 1.5mm] (cap4) at (e1.north) {\small{4?}};
			\node [yshift = 1.5mm] (cap4) at (e2.north) {\small{4?}};
			\node [yshift = 1.5mm] (cap5) at (e3.north) {\small{4?}};
			\node [yshift = 1.5mm] (cap5) at (e4.north) {\small{4?}};
			\node [yshift = 1.5mm] (cap6) at (n8.north) {\small{4?}};
			\node [yshift = 1.5mm] (cap7) at (n9.north) {\small{4?}};
		\end{tikzpicture}
	\end{minipage}%
	\begin{minipage}{.33\textwidth}
		\begin{tikzpicture}[node distance=\nodedistance]
			\proofnodeBW{root};
			\proofnodeBW[above left of = root,xshift=-\nodedistance]{n7};
			\proofnodeBW[above right of = n7,xshift=\nodedistance]{n6};
			\proofnodeBW[above left of = n7]{n3};
			\proofnodeBW[above right of = root,xshift=\nodedistance]{n12};
			\proofnodeBW[above right of = n12]{n10};
			\withchildrenBW{n3}{n1}{n2};
			\withchildrenBW{n10}{n8}{n9};
			\proofnodeBW[above right of = n2]{n4};
			\proofnodeBW[above left of = n8]{n5};
			\drawchildren{n6}{n4}{n5};
			\withchildrenBW{n4}{e1}{e2};
			\withchildrenBW{n5}{e3}{e4};
			\drawchildren{root}{n7}{n12};
			\drawchildren{n7}{n3}{n6};
			\drawchildren{n12}{n6}{n10};
			\blacknode{n3};
			\blacknode{n8};
			\node [yshift = 2mm] (cap1) at (n1.north) {\small{1}};
			\node [yshift = 2mm] (cap2) at (n2.north) {\small{2}};
			\node [yshift = 2mm] (cap3) at (n3.north) {\small{3}};
			\node [yshift = 2mm] (cap8) at (n8.north) {\small{4}};
		\end{tikzpicture}
	\end{minipage}%
	\begin{minipage}{.33\textwidth}
	\begin{tikzpicture}[node distance=\nodedistance]
			\proofnodeBW{root};
			\proofnodeBW[above left of = root,xshift=-\nodedistance]{n7};
			\proofnodeBW[above right of = n7,xshift=\nodedistance]{n6};
			\proofnodeBW[above left of = n7]{n3};
			\proofnodeBW[above right of = root,xshift=\nodedistance]{n12};
			\proofnodeBW[above right of = n12]{n10};
			\withchildrenBW{n3}{n1}{n2};
			\withchildrenBW{n10}{n8}{n9};
			\proofnodeBW[above right of = n2]{n4};
			\proofnodeBW[above left of = n8]{n5};
			\drawchildren{n6}{n4}{n5};
			\withchildrenBW{n4}{e1}{e2};
			\withchildrenBW{n5}{e3}{e4};
			\drawchildren{root}{n7}{n12};
			\drawchildren{n7}{n3}{n6};
			\drawchildren{n12}{n6}{n10};
			\blacknode{n3};
			\blacknode{n10};
			\blacknode{n4};
			\blacknode{n5};
			\blacknode{e3};
			\blacknode{e4};
			\node [yshift = 2mm] (cap1) at (n1.north) {\small{1}};
			\node [yshift = 2mm] (cap2) at (n2.north) {\small{2}};
			\node [yshift = 2mm] (cap3) at (n3.north) {\small{3}};
			\node [yshift = 2mm] (cap4) at (n8.north) {\small{4}};
			\node [yshift = 2mm] (cap5) at (n9.north) {\small{5}};
			\node [yshift = 2mm] (cap6) at (n10.north) {\small{6}};
			\node [yshift = 2mm] (cap7) at (e1.north) {\small{7}};
			\node [yshift = 2mm] (cap8) at (e2.north) {\small{8}};
			\node [yshift = 2mm] (cap8) at (n4.north) {\small{9}};
			\node [yshift = 2mm] (cap7) at (e3.north) {\small{10}};
			\node [yshift = 2mm] (cap8) at (e4.north) {\small{11}};
			\node [yshift = 2mm] (cap8) at (n5.north) {\small{12}};
		\end{tikzpicture}
	\end{minipage}%
	}
	\caption{Top-Down Pebbling}
	\label{fig:TDP}
\end{figure}

Top-Down Pebbling often ends up finding a sub-optimal pebbling strategy regardless of the heuristic used. Consider the graph shown in Figure \ref{fig:TDP} and suppose that top-down pebbling has already pebbled the initial sequence of nodes $(1,2,3)$. 
For a greedy heuristic that only has information about pebbled nodes, their premises and children, all nodes marked with $4?$ are considered equally worthy to pebble next.
Suppose the node marked with $4$ in the middle graph is chosen to be pebbled next.
Subsequently, pebbling $5$ opens up the possibility to remove a pebble after the next move, which is to pebble $6$.
After that only the middle subgraph has to be pebbled. No matter in which order this is done, the strategy will use six pebbles at some point. 
One example sequence and the point where six pebbles are used are shown in the rightmost picture in Figure \ref{fig:TDP}.

\label{example:TDPIssue}
\end{example}

\subsection{Bottom-Up Pebbling}

\algo{Bottom-Up} Pebbling (Algorithms \ref{algo:BUpebbling} and \ref{algo:visit}) constructs a topological order of a proof $\varphi$ while traversing it from its root node to its axioms. The algorithm constructs the order by visiting nodes and their premises recursively. At every node $\n$ the order in which the premises of $\n$ are visited is decided heuristically. After visiting the premises, $n$ is added to the current sequence of nodes.
Since axioms do not have any premises, there is no recursive call for axioms and these nodes are simply added to the sequence. The recursion is started by a call to visit the root.
Since all proof nodes are ancestors of the root, the recursive calls will eventually visit all nodes once and a topological total order will be found.
Bottom-Up Pebbling corresponds to how the apply function $\ap(.)$ was defined in Section \ref{sec:pebblingchecking} with the addition of a visit order of the premises for each node and the omission of calls to previously processed nodes.

\SetKwFunction{KwVisit}{visit}

\begin{algorithm}[h]
  \KwIn{proof $\varphi$ with root node $r$}
  \KwOut{sequence of nodes $S$ representing a topological order $\prec$ of $\varphi$}
  \BlankLine

	$S = ()$\; \tcp*[f]{the empty sequence}\\
	$V = \emptyset$\;
	\Return \KwVisit{$\varphi$,$r$,$V$,$S$}\;

  \caption[.]{\FuncSty{Bottom-Up Pebbling}}
  \label{algo:BUpebbling}
\end{algorithm}

\begin{algorithm}[h]
  \KwIn{proof $\varphi$}
	\KwIn{node $\n$}
	\KwIn{set of visited nodes $V$} 
	\KwIn{initial sequence of nodes $S$}
  \KwOut{sequence of nodes}
	
	$V_1 = V \cup \{\n\}$;
	$N = \Premises{\n}{\varphi} \setminus V$; \tcp*[f]{Only unprocessed premises are visited} \\
	$S_1 = S$\;
	
  \While{$N$ is not empty}{
    choose $p \in N$ heuristically;
		$N = N \setminus p$\;
		$S_1 = S_1 ::: visit(\varphi,p,V,S)$; \tcp*[f]{$:::$ is the concatenation of sequences}
  }
	
	\Return $S_1 ::: (\n)$\;
	
  \caption[.]{\FuncSty{visit}}
  \label{algo:visit}
\end{algorithm}

%\newcommand{\nodedistance2}{0.6cm}
\begin{example}
Figure \ref{fig:BUP} shows part of an execution of \algo{Bottom-Up} Pebbling on the same proof as presented in Figure \ref{fig:TDP}.
Nodes chosen by the heuristic, to be processed before the respective other premise, are marked dashed. 
Suppose that similarly to the Top-Down Pebbling scenario, nodes have been chosen in such a way that the initial pebbling sequence is $(1,2,3)$.
However, the choice of where to go next is predefined by the dashed nodes. 
Consider the dashed child of node $3$. 
Since $3$ has been completely processed, the other premise of its dashed child is visited next. 
The result is that node the middle subgraph is pebbled while only one external node is pebbled, while it have been two in the Top-Down scenario. 
At no point more than five pebbles will be used for pebbling the root node, which is shown in the bottom right picture of the figure. This is independently of the heuristic choices.

Not that this example underlines the point of two observations: pebbling choices should be made local and hard subgraphs should be pebbled early.

\begin{figure}[tb]
	\makebox[\textwidth][c]{
		\begin{minipage}{0.4\textwidth}
			\begin{tikzpicture}[node distance=\nodedistanceThree]
				\proofnodeBW{root};
				\proofnodeBW[above left of = root,xshift=-\nodedistanceThree]{n7};
				\proofnodeBW[above right of = n7,xshift=\nodedistanceThree]{n6};
				\proofnodeBW[above left of = n7]{n3};
				\proofnodeBW[above right of = root,xshift=\nodedistanceThree]{n12};
				\proofnodeBW[above right of = n12]{n10};
				\withchildrenBW{n3}{n1}{n2};
				\withchildrenBW{n10}{n8}{n9};
				\proofnodeBW[above right of = n2]{n4};
				\proofnodeBW[above left of = n8]{n5};
				\drawchildren{n6}{n4}{n5};
				\withchildrenBW{n4}{e1}{e2};
				\withchildrenBW{n5}{e3}{e4};
				\drawchildren{root}{n7}{n12};
				\drawchildren{n7}{n3}{n6};
				\drawchildren{n12}{n6}{n10};
				\waitingnode{n7};
				\waitingnode{root};
				\blacknode{n3};
				\node [yshift = 2mm] (cap1) at (n1.north) {\small{1}};
				\node [yshift = 2mm] (cap2) at (n2.north) {\small{2}};
				\node [yshift = 2mm] (cap3) at (n3.north) {\small{3}};
			\end{tikzpicture}
		\end{minipage}%
		\begin{minipage}{0.4\textwidth}
			\begin{tikzpicture}[node distance=\nodedistance]
				\proofnodeBW{root};
				\proofnodeBW[above left of = root,xshift=-\nodedistanceThree]{n7};
				\proofnodeBW[above right of = n7,xshift=\nodedistanceThree]{n6};
				\proofnodeBW[above left of = n7]{n3};
				\proofnodeBW[above right of = root,xshift=\nodedistanceThree]{n12};
				\proofnodeBW[above right of = n12]{n10};
				\withchildrenBW{n3}{n1}{n2};
				\withchildrenBW{n10}{n8}{n9};
				\proofnodeBW[above right of = n2]{n4};
				\proofnodeBW[above left of = n8]{n5};
				\drawchildren{n6}{n4}{n5};
				\withchildrenBW{n4}{e1}{e2};
				\withchildrenBW{n5}{e3}{e4};
				\drawchildren{root}{n7}{n12};
				\drawchildren{n7}{n3}{n6};
				\drawchildren{n12}{n6}{n10};
				\waitingnode{n7};
				\blacknode{n3};
				\waitingnode{n6};
				\waitingnode{root};
				\node [yshift = 2mm] (cap1) at (n1.north) {\small{1}};
				\node [yshift = 2mm] (cap2) at (n2.north) {\small{2}};
				\node [yshift = 2mm] (cap3) at (n3.north) {\small{3}};
			\end{tikzpicture}
		\end{minipage}%
		}
		\makebox[\textwidth][c]{
		\begin{minipage}{0.4\textwidth}
			\begin{tikzpicture}[node distance=\nodedistanceThree]
				\proofnodeBW{root};
				\proofnodeBW[above left of = root,xshift=-\nodedistanceThree]{n7};
				\proofnodeBW[above right of = n7,xshift=\nodedistanceThree]{n6};
				\proofnodeBW[above left of = n7]{n3};
				\proofnodeBW[above right of = root,xshift=\nodedistanceThree]{n12};
				\proofnodeBW[above right of = n12]{n10};
				\withchildrenBW{n3}{n1}{n2};
				\withchildrenBW{n10}{n8}{n9};
				\proofnodeBW[above right of = n2]{n4};
				\proofnodeBW[above left of = n8]{n5};
				\drawchildren{n6}{n4}{n5};
				\withchildrenBW{n4}{e1}{e2};
				\withchildrenBW{n5}{e3}{e4};
				\drawchildren{root}{n7}{n12};
				\drawchildren{n7}{n3}{n6};
				\drawchildren{n12}{n6}{n10};
				\blacknode{n7};
				\blacknode{n6};
				\waitingnode{root};
				\node [yshift = 2mm] (cap1) at (n1.north) {\small{1}};
				\node [yshift = 2mm] (cap2) at (n2.north) {\small{2}};
				\node [yshift = 2mm] (cap3) at (n3.north) {\small{3}};
				\node [yshift = 2mm] (cap4) at (e1.north) {\small{5}};
				\node [yshift = 2mm] (cap5) at (e2.north) {\small{4}};
				\node [yshift = 2mm] (cap6) at (n4.north) {\small{6}};
				\node [yshift = 2mm] (cap7) at (e3.north) {\small{7}};
				\node [yshift = 2mm] (cap7) at (e4.north) {\small{8}};
				\node [yshift = 2mm] (cap7) at (n5.north) {\small{9}};
				\node [yshift = 2mm] (cap7) at (n6.north) {\small{10}};
			\end{tikzpicture}
		\end{minipage}%
		\begin{minipage}{0.4\textwidth}
			\begin{tikzpicture}[node distance=\nodedistanceThree]
				\proofnodeBW{root};
				\proofnodeBW[above left of = root,xshift=-\nodedistanceThree]{n7};
				\proofnodeBW[above right of = n7,xshift=\nodedistanceThree]{n6};
				\proofnodeBW[above left of = n7]{n3};
				\proofnodeBW[above right of = root,xshift=\nodedistanceThree]{n12};
				\proofnodeBW[above right of = n12]{n10};
				\withchildrenBW{n3}{n1}{n2};
				\withchildrenBW{n10}{n8}{n9};
				\proofnodeBW[above right of = n2]{n4};
				\proofnodeBW[above left of = n8]{n5};
				\drawchildren{n6}{n4}{n5};
				\withchildrenBW{n4}{e1}{e2};
				\withchildrenBW{n5}{e3}{e4};
				\drawchildren{root}{n7}{n12};
				\drawchildren{n7}{n3}{n6};
				\drawchildren{n12}{n6}{n10};
				\blacknode{n7};
				\blacknode{n6};
				\blacknode{n8};
				\blacknode{n9};
				\blacknode{n10};
				\waitingnode{root};
				\waitingnode{n12};
				\node [yshift = 2mm] (cap1) at (n1.north) {\small{1}};
				\node [yshift = 2mm] (cap2) at (n2.north) {\small{2}};
				\node [yshift = 2mm] (cap3) at (n3.north) {\small{3}};
				\node [yshift = 2mm] (cap4) at (e1.north) {\small{5}};
				\node [yshift = 2mm] (cap5) at (e2.north) {\small{4}};
				\node [yshift = 2mm] (cap6) at (n4.north) {\small{6}};
				\node [yshift = 2mm] (cap7) at (e3.north) {\small{7}};
				\node [yshift = 2mm] (cap7) at (e4.north) {\small{8}};
				\node [yshift = 2mm] (cap7) at (n5.north) {\small{9}};
				\node [yshift = 2mm] (cap7) at (n6.north) {\small{10}};
				\node [yshift = 2mm] (cap7) at (n8.north) {\small{11}};
				\node [yshift = 2mm] (cap7) at (n9.north) {\small{12}};
				\node [yshift = 2mm] (cap7) at (n10.north) {\small{13}};
			\end{tikzpicture}
		\end{minipage}%
			}
		\caption{Bottom-Up Pebbling}
		\label{fig:BUP}
\end{figure}
\label{example:BUP}
\end{example}



\subsection{Remarks about Top-Down and Bottom-Up Pebbling} %or: Which way to go?
\label{sec:TDvsBU}

In principle every topological order of a given proof can be constructed using Top-down or Bottom-up Pebbling. Both algorithms traverse the proof only once and have linear run-time in the proof length (assuming that the heuristic choice requires constant time). Example \ref{example:TDPIssue} shows a situation where \algo{Top-Down} Pebbling may pebble a node that is far away from the previously pebbled nodes. This results in a sub-optimal pebbling strategy.
As discussed in Example \ref{example:BUP}, \algo{Bottom-Up} Pebbling is more immune to this non-locality issue, because queuing up the processing of premises enforces local pebbling. This suggests that \algo{Bottom-Up} is better than \algo{Top-Down}, which is confirmed by the experiments in Section \ref{sec:exp}.



%%%%%%%%%%%%%%%%%%%%%%%%%%%%%%%%%%%%%%%%%
%%% Heuristics %%%%%%%%%%%%%%%%%%%%%%%%%%
%%%%%%%%%%%%%%%%%%%%%%%%%%%%%%%%%%%%%%%%%
%\chapter{Heuristics}
%\label{ch:heuristics}
%%%%%%%%%%%%%%%%%%%%%%%%%%%%%%%%%%%%%%%%%

\section{Heuristics}
\label{sec:heuristics}

Both pebbling algorithms described in the previous section have to choose a node $h(N)$ out of a set $N$ at some point, where $h$ is a \emph{heuristic} function. For \algo{Top-Down} Pebbling, $N$ is the set of pebbleable nodes, and for \algo{Bottom-Up} Pebbling, $N$ is the set of premises of a node. Every heuristic $h$ considered here uses an auxiliary \emph{node evaluation} function $e_h$ that maps nodes to elements of a totally ordered set $S_h$. The chosen node $h(N)$ is then simply defined as $\mathit{argmax}_{\n \in N} e_h(\n)$ (i.e. $h(N)$ is a node in $N$ such that for all nodes $\n$ in N, $e_h(h(N)) \geq e_h(\n)$).

\subsection{Number of Children Heuristic (``$Ch$'')}
\label{sec:children}
This heuristic uses the number of children of a node $\n$: $e_h(\n) = |\Children{\n}{\varphi}|$ and $S_h = \mathbb{N}$.
The intuitive motivation for this heuristic is that nodes with many children will require many pebbles, and subproofs containing nodes with many children will tend to be more spacious. Example \ref{example:hardfirst} shows why it is a good idea to process spacious subproofs first.

\begin{example}
Figure \ref{fig:SpaciousFirst} shows a simple proof $\varphi$ with two subproofs $\varphi_1$ (left branch) and $\varphi_2$ (right branch). As shown in the leftmost diagram, assume $\pspace{\varphi_1}{<^1_T} = 4$ and $\pspace{\varphi_2}{<^2_T} = 5$.
After pebbling one of the subproofs, the pebble on its root node has to be kept there until the root of the other subproof is also pebbled. Only then the root node can be pebbled. Therefore, $\pspace{\varphi}{\prec} = \pspace{\varphi_j}{<^j_T} + 1$ where $\prec = <^i_T ::: <^j_T$ and $i$ and $j$ are the indexes of the subproofs pebbled, respectively first and last. 
Choosing to pebble the least spacious subproof $\varphi_1$ first results in $\pspace{\varphi}{\prec} = 6$, while pebbling the most spacious one first gives $\pspace{\varphi}{\prec} = 5$.
This is a simplified situation. The two subproofs do not share nodes. Pebbling one of them does not influence the pebble number of the other.

\begin{figure}[h]
	\usetikzlibrary{shapes.geometric}
	\tikzset{
    triangle/.style={
        draw,
        shape border rotate=180,
        regular polygon,
        regular polygon sides=3,
        node distance=\nodedistanceTwo,
        %minimum height=4em,
				minimum width= 1.5cm
			}
	}
	\makebox[\textwidth][c]{
	\begin{minipage}{.3\textwidth}
		\begin{tikzpicture}[node distance=\nodedistanceTwo]
			\node[circle, draw, anchor=mid](root) {?};
			\addchildrenBW{root}{n1}{n2};
			\node[triangle, above of = n1, yshift = -4mm]{4};
			\node[triangle, above of = n2, yshift = -4mm]{5};
			\whitenode{n1};
			\whitenode{n2};
			\drawchildren{root}{n1}{n2};
		\end{tikzpicture}
	\end{minipage}%
		\begin{minipage}{.3\textwidth}
		\begin{tikzpicture}[node distance=\nodedistanceTwo]
			\node[circle, draw, anchor=mid](root) {6};
			\addchildrenBW{root}{n1}{n2};
			\node[triangle, above of = n1, yshift = -4mm]{ };
			\node[triangle, above of = n2, yshift = -4mm]{5};
			\blacknode{n1};
			\whitenode{n2};
			\drawchildren{root}{n1}{n2};
		\end{tikzpicture}
	\end{minipage}%
		\begin{minipage}{.3\textwidth}
		\begin{tikzpicture}[node distance=\nodedistanceTwo]
			%\proofnodeBW{root};
			\node[circle, draw, anchor=mid](root) {5};
			\addchildrenBW{root}{n1}{n2};
			\node[triangle, above of = n1, yshift = -4mm]{4};
			\node[triangle, above of = n2, yshift = -4mm]{ };
			\whitenode{n1};
			\blacknode{n2};
			\drawchildren{root}{n1}{n2};
		\end{tikzpicture}
	\end{minipage}%
	}
	\caption{Spacious subproof first}
	\label{fig:SpaciousFirst}
\end{figure}
\label{example:hardfirst}
\end{example}

\subsection{Last Child Heuristic (``$Lc$'')}
\label{sec:lastchild}

As discussed in Section \ref{sec:pebblingchecking} in the proof of Theorem \ref{theorem:canonical}, the best moment to unpebble a node $\n$ is as soon as its last child w.r.t. a topological order $\prec$ is pebbled. 
This insight can be used for a heuristic that prefers nodes that are last children of other nodes. Pebbling a node that allows another one to be unpebbled is always a good move. 
The current number of used pebbles (after pebbling the node and unpebbling one of its premises) does not increase; 
it might even decrease, if more than one premise can be unpebbled.
For determining the number of premises of which a node is the last child, the proof has to be traversed once, using some topological order $\prec$.
Before the traversal, $e_h(\n)$ is set to 0 for every node $\n$. During the traversal $e_h(\n)$ is increased by 1, if $\n$ is the last child of the currently processed node w.r.t. $\prec$. For this heuristic $S_h = \mathbb{N}$.
To some extent, this heuristic is paradoxical: $\n$ may be the last child of a node $\n'$ according to $\prec$, but pebbling it early may result in another topological order $<^*_T$ according to which $\n$ is not the last child of $\n'$.
Nevertheless, sometimes the proof structure ensures that some nodes are the last child of another node irrespective of the topological order. An example is shown in Figure \ref{fig:forcedLC}, where the dashed line denotes a recursive predecessor relationship and the bottommost node is the last child of the top right node in every topological order.

\begin{figure}[h]
	\centering{
	\begin{tikzpicture}[node distance=1cm]
		\proofnodeBW{n4};
		\proofnodeBW[above left of = n4]{n3};
		
		%\proofnodeBW[above left of = n3]{n1};
		
		\proofnodeBW[above right of = n3]{n2};
		%\withchildrenBW{n3}{n1}{n2};
		%\drawchildren{n4}{n3}{n2};
		%\node[above left of=n3] (emptynode) {};
		%\draw[->,thick,cap=round,dashed] (n3) to (emptynode);
		\draw[->,thick,cap=round,dashed] (n3) to (n2);
		\draw[->,thick,cap=round] (n4) to (n2);
		\draw[->,thick,cap=round,dashed] (n4) to (n3);
		
	\end{tikzpicture}
	}
	\caption{Bottommost node as necessary last child of right topmost node}
	\label{fig:forcedLC}
\end{figure}


\subsection{Node Distance Heuristic (``$Dist(r)$'')}
\label{sec:distance}

In Example \ref{example:topdown} and Section \ref{sec:TDvsBU} it has been noted that \algo{Top-Down} Pebbling may perform badly if nodes that are far apart are selected. The Node \algo{Distance} Heuristic prefers to pebble nodes that are close to pebbled nodes. It does this by calculating spheres with a radius up to the parameter $r$ around nodes.
A sphere $K_r^{G}(\n)$ with radius $r$ around the node $\n$ in the graph $G = (V,E)$ is the set $\{p \in V \mid \text{ there are at most } r \text{ edges between } p \text{ and } \n\}$. The direction of edges is not considered. The heuristic uses the following functions based on the spheres:
\begin{align*}
d(\n) &\defeq -min\{r \mid K_r^{G}(\n)\text{ contains a pebbled node}\}\\
	s(\n) &\defeq |K_{-d(\n)}^G(\n)|\\
	l(\n) &\defeq max_{<_N}K_{-d(\n)}^G(\n)\\
	e_h(\n) &\defeq (d(\n),s(\n),l(\n))
\end{align*}
where $<_N$ denotes the total order on the initial sequence of pebbled nodes $N$, i.e. nodes in spheres that were pebbled later are preferred.
So $S_h = \mathbb{Z} \times \mathbb{N} \times P$ together with the lexicographic order using, respectively, the natural smaller relation $<$ on $\mathbb{Z}$ and $\mathbb{N}$ and $<_N$ on $N$. The spheres $K_r(\n)$ can grow exponentially in $r$. Therefore the maximum radius has to be limited and if no pebbled node is found within in any sphere with this radius, another heuristic has to be used.


\subsection{Decay Heuristics (``$Dc(h_u,\gamma,d,com)$'') }
\label{sec:decay}
\algo{Decay} Heuristics denote a family of meta heuristics. 
The idea is to not only use the evaluation of a single node, but also to include the evaluations of its premises.
Such a heuristic has four parameters: an underlying heuristic $h_u$ defined by an evaluation function $e_u$ together with a well ordered set $S_u$, a decay factor $\gamma \in \mathbb{R}^+ \cup \{0\}$, a recursion depth $d \in \mathbb{N}$ and a combining function $com: S_u^n \rightarrow S_u$ for $n \in \mathbb{N}$.\\
The resulting heuristic node evaluation function $e_h$ is defined with the help of the recursive function $rec$:
\begin{align*}
	rec(\n,0) &\defeq e_u(\n) \\
	rec(\n,k) &\defeq e_u(\n) + com(rec(p_1,k-1),\ldots,rec(p_n,k-1)) * \gamma\\
	& \text{ where } \Premises{\n}{\varphi} = \{p_1,\ldots,p_n\}\\
	% & \text{ and } k \in \{1,\ldots,d\} \\
	%rec(\n,k) &\defeq com(h_u(\n), rec(p_1,k-1)*\gamma,\ldots,rec(p_{n-1},k-1) * \gamma)\\ %might be better - I will try this in the experiments
	e_h(\n) &\defeq rec(\n,d) &
\end{align*}



%%%%%%%%%%%%%%%%%%%%%%%%%%%%%%%%%%%%%%%%%
%%% Experiments %%%%%%%%%%%%%%%%%%%%%%%%%
%%%%%%%%%%%%%%%%%%%%%%%%%%%%%%%%%%%%%%%%%
%\chapter{Experiments}
%\label{ch:experiments}
%%%%%%%%%%%%%%%%%%%%%%%%%%%%%%%%%%%%%%%%%

%\section{Experiments} 
\label{sec:experiments}


All the pebbling algorithms and heuristics described in the previous sections have been implemented in the hybrid functional and object-oriented programming
language Scala (\url{www.scala-lang.org}) as part of the \skeptik library for proof compression (\url{github.com/Paradoxika/Skeptik}) \cite{Skeptik}.
In order to evaluate them, experiments were executed\footnote{The Vienna Scientific Cluster VSC\nobreakdash-2 
(\url{http://vsc.ac.at/}) was used.} on four disjoint sets of proof benchmarks (Table \ref{tab:benchmarks}). Two of them contain proofs produced by the SAT-solver \texttt{PicoSAT} \cite{Biere_picosatessentials} on unsatisfiable benchmarks from the SATLIB (\url{www.satlib.org/benchm.html}) library. The proofs\footnote{SAT proofs: \url{www.logic.at/people/bruno/Experiments/2014/Pebbling/tc-proofs.zip}} are in the TraceCheck format, which is one of the three formats accepted at the \emph{Certified Unsat} track of the SAT-Competition.
The other two benchmark sets contain proofs produced by the SMT-solver {\veriT} (\url{www.verit-solver.org}) 
on unsatisfiable problems from the SMT-Lib (\url{www.smtlib.org}). These proofs\footnote{SMT proofs: \url{www.logic.at/people/bruno/Experiments/2014/Pebbling/smt-proofs.zip}} are in a proof format that resembles SMT-Lib's problem format and they were translated into pure resolution proofs by considering every non-resolution inference as an axiom.

Figure \ref{fig:SpaceVSLength} relates for each proof the smallest space measure obtained by all algorithms tested on the respective proof and its length in number of nodes. Note that the y-axis, showing the space measures, scale is lower by a factor of 100 than the scale of the x-axis, which displays the proof lengths. On average the smallest space measure of a proof is 44,1 times smaller than its length. This shows the impact that the usage of deletion information together with well constructed topological orders can have. When these techniques are used, on average 44,1 times less memory is required for checking a proof.

\begin{figure}
	\centering
	\includegraphics[scale=0.4]{Figures/length_vs_space_2.png}
	\caption{Best space measure compared to proof length}
	\label{fig:SpaceVSLength}
\end{figure}

\begin{figure}
	\centering
	\includegraphics[scale=0.4]{Figures/TD_vs_BU-scatter_min.png}
	\caption{Spaces obtained with best \algo{Bottom-Up} and \algo{Top-Down} heuristics}
	\label{fig:BUvsTD}
\end{figure}


\begin{equation} \label{eq:space}
  \mathit{performance}(f, G, P) = \frac{1}{|P|} * \sum_{\varphi \in P}{\left( 1 -
    \frac{
      s(\varphi,f(\varphi))
    }{
        \mathit{avg}_{g\in G}{s(\varphi,g(\varphi))}
    } \right)
  }
\end{equation}

\begin{table}[tb]
	\centering
	\setlength{\tabcolsep}{8pt}
	\begin{tabular}{|l|c|c|c|}
		\hline
		\textbf{Name} & \textbf{Number of proofs} & \textbf{Maximum length} & \textbf{Average length} \\ 
		\hline \hline
		TRC1 & 2239 & 90756   & 5423   \\ \hline
		TRC2 & 215	& 1768249 & 268863 \\ \hline
    SMT1 & 4187 & 2241042 & 103162 \\ \hline
    SMT2 & 914  & 120075  & 5391  \\ 
		\hline   
	\end{tabular}
	\caption{Proof benchmark sets}
	\label{tab:benchmarks}
\end{table}

\newcommand{\cHline}{\\[-2.5ex] \hline \\[-2.5ex]}
\begin{table}[tb]
\centering
\setlength{\tabcolsep}{8pt}
\begin{tabular}{|l|c|c|c|c|c|c}
\hline
\textbf{Algorithm} & \multicolumn{4}{c|}{\textbf{Relative Performance} (\%)} & \textbf{Speed}\\ 
Heuristic:Method & \textbf{SMT1} & \textbf{SMT2} & \textbf{TRC1} & \textbf{TRC2} & (nodes/ms)\\ 
\hline\hline
Ch:BU & 19,53 & -15,79 & 20,48 & \textbf{88,57} & \textbf{88,55} \\ 
Ch:TD & \textbf{-22,07} & 8,29 & -48,33 & -67,12 & 0,30 \\ \cHline
Lc:BU & \textbf{23,42}& 36,69 & 21,47 & 88,55 & 84,43 \\ 
Lc:TD & -20,88 & 14,20 & -64,07 & \textbf{-110,00} & 1,87 \\ \cHline
Dist(1):BU & & -15,72 & 19,74 & & 21,23 \\ 
Dist(1):TD & & -67,52 & -71,21 & & 0,63 \\ 
Dist(3):BU & & -50,27 & 19,95 & & 0,54\\ 
Dist(3):TD & & \textbf{-74,90} & \textbf{-74,09} & & \textbf{0,08}\\ \cHline
Dc(LC,0.5,1,avg):BU & & 37,39 & 21,83 & & 47,70\\ 
Dc(LC,0.5,7,avg):BU & & 37,78 & 22,05 & & 14,01 \\
Dc(LC,3,1,avg):BU & & 36,86 & 22,02 & & 63,97\\
Dc(LC,3,7,avg):BU & & 34,69 & \textbf{22,55} & & 15,31 \\ 
Dc(LC,0.5,1,max):BU & & 37,31 & 21,76 & & 47,03 \\
Dc(LC,0.5,7,max):BU & & 37,89 & 21,94 & & 15,26 \\
Dc(LC,3,1,max):BU & & 37,33 & 21,79 & & 64,43 \\
Dc(LC,3,7,max):BU & & \textbf{37,96} & 22,13 & & 15,34 \\
%\bottomrule
\hline
\end{tabular}
\caption{Experimental results}
\label{tab:results}
\end{table}

\noindent
Table \ref{tab:results} shows that \algo{Bottom-Up} algorithms construct topological orders with much smaller space measures than \algo{Top-Down} algorithms. This fact is visualized in Figure \ref{fig:BUvsTD}, where each dot represents a proof $\varphi$ and the $x$ and $y$ coordinates show the space of $\varphi$ with the topological orders found by, respectively, the best \algo{Top-Down} and \algo{Bottom-Up} algorithms for $\varphi$. Some other heuristics (not described in this paper) aimed at improving \algo{Top-Down} Pebbling were tested on small benchmark sets, but none showed promising results.

Furthermore, \algo{Bottom-Up} algorithms are also much faster, as can be seen in the last column of Table \ref{tab:results}. This is so because they require fewer comparisons in their heuristic choices. For \algo{Bottom-Up} algorithms, the set $N$ of possible choices consists of the premises of a single node only, and usually $|N| \in O(1)$ (e.g. for a binary resolution proof, $N \leq 2$ always). On the other hand, the set $N$ of currently pebbleable nodes, from which \algo{Top-Down} algorithms must choose, is large (e.g. for a perfect binary tree with $2n -1$ nodes, initially $|N| = n$). For some heuristics, \algo{Top-Down} algorithms could be made more efficient by using, instead of a set, an ordered sequence of pebbleable nodes together with their memorized heuristic evaluations.

Using the \algo{Distance} Heuristic has a severe impact on the speed, which decreases rapidly as the maximum radius increases. With a radius equal to 5, only a few small proofs were processed in a reasonable amount of time.

As expected, the \algo{Decay} Heuristic does improve the results of the underlying heuristic. Note that because of the relative nature of the performance measure and the poor performance of the \algo{Top-Down} algorithms, small performance differences can still be significant. Nevertheless, the performance improvement comes at a high cost in speed.



\section{Conclusions}



\bibliographystyle{splncs}
\bibliography{biblio}


\end{document}

% vim: tw=100
